
\vspace{0.5cm} \textbf{2022-07-03: 8.40 $\rightarrow$ 11.40}

Heute habe ich den Tag genutzt, um mir ein allgemeines Verständnis vom
Framework Flutter zu erschaffen. Hierbei schaute ich mir
verschiedene Dokumentationen bzw. Videos an, um die Grundstruktur von
Flutter zu verstehen. Schlussendlich kam ich zu dem Punkt mir das
Framework herunterzuladen, allerdings blieb ich bei der Installation bei
einem Punkt hängen und entschied mich die Installation erst morgen
anzugehen.

\vspace{0.5cm} \textbf{2022-07-04: 11.20 $\rightarrow$ 14.20}

Heute wurde das Framework Flutter installiert. Allerdings war es
zunächst nicht möglich, die Flutter-App in VsCode zu kompilieren. Zum Glück
fand ich in einem Forum eine Lösung und konnte somit die vorzeige App
kompilieren. Danach musste ich verschiedene Extension herunterladen und den
Editor für Flutter richtig einstellen. Der restliche Tag wurde noch
damit verbracht sich Videos über Flutter anzusehen

\vspace{0.5cm} \textbf{2022-07-05: 7.00 $\rightarrow$ 12.00}

Den heutigen Tag habe ich damit verbracht, die Programmiersprache Dart zu
verstehen und wie man diese mit dem Framework Flutter verwendet.
Zunächst habe ich verschiedene Widgets erstellt und diese mit einer
Logik in Dart getestet.Es wurde eine Testversion des Logins. Abends hatten meine Kollegen und ich noch ein Gespräch, wie
wir das Projekt angehen bzw. wer was macht.

\vspace{0.5cm} \textbf{2022-07-06: 12.00 $\rightarrow$ 18.00}

Heute habe ich mir verschiedene Videos angeschaut, wie man Passwörter
verschlüsselt am Smartphone speichern kann (Login „Remember me``
Funktion). Allerdings trat ich auf ein Problem, da ich verschiedene
Funktionen in Dart nicht verstand, die dazu benötigt waren. Schließlich
verstand ich diese  Danach nutzte ich die Zeit, um den Zugriff auf Rest-APIs in Flutter zu erm\"ogliochen. Am Abend hatte ich mit meinen Kollegen noch ein Meeting, um
ein Konzept für unser Projekt zu gestalten. Zunächst wurde ein
allgemeines yuml-File entworfen, welches den Lifecycle des Projekts
darstellen soll. Danach schrieb jeder, was sein Teilbereich überhaupt
alles implementieren soll

\vspace{0.5cm} \textbf{2022-07-07: 7.00 $\rightarrow$ 11.00}

Heute schrieb ich eine Test-App, die zu jeder ID (z. B. Tresor Karte) von
einem JSON File (welches man über eine REST-API bekommt) eine
dementsprechende Widget-Card erstellen soll. Zunächst überlegte ich mir
einen Algorithmus und zeichnete diesen auf einem Blatt Papier auf. Nach
zahlreichen Denkfehlern schaffte ich es allerdings den Parser zum Laufen
zu bekommen. Danach nutzte ich die Zeit und schaute mir verschiedene
Videos, wie meine größeren Projekte mit Flutter strukturieren soll.
Dementsprechend erstellte ich eine angepasste Struktur für unser Projekt
und teilte es Ben mit, wie diese funktioniert.

\vspace{0.5cm} \textbf{2022-07-07: 11.00 $\rightarrow$ 13.00}

Es wurde eine Mockup für die kommende Anwendung erstellt

\vspace{0.5cm} \textbf{2022-08-16: 7.40 $\rightarrow$ 13.10}

Nach einmonatiger Pause aufgrund eines Praktikums hatte ich heute mit
Herrn Zöchmann ein Meeting, um über den aktuellen Stand des Projektes zu
reden. Es wurden Änderungen bei der Ordnerstruktur durchgeführt und
eine Fehlerbehandlung umgesetzt. Da ich ca. ein Monat nicht beim Projekt
weiterarbeiten konnte, musste ich mich bei meinem aktuellen Fortschritt
wieder einlesen. Danach fing ich mit der Registrierungsseite an. Ich
entschied mich, eine Progressbar zu programmieren, die das eingegebene
Passwort bewertet. Da ich kein Paket aufgrund der Sicherheit verwenden
wollte, programmierte ich die Logik selbst. Allerdings hatte ich kleine
Komplikationen (Regex), die mich etwas Zeit kosteten. Schlussendlich
funktionierte dann der Passwortbewerter

\vspace{0.5cm} \textbf{2022-08-16: 15.00 $\rightarrow$ 16.00}

Am Nachmittag wurde eine Stunde bei der Registrierungsseite
weitergearbeitet. Es wurden Fehler ausgebessert und es wurde ein
Passwort wiederholen Textfeld hinzugefügt. Weiters wurde bei der
Gesamtlogik etwas geändert

\vspace{0.5cm} \textbf{2022-08-17: 9.00 $\rightarrow$ 11.15}

Am Vormittag hatte ich mit Herrn Zöchmann ein Meeting um zu kl\"aren welche Daten
die API uns zur Verfügung stellen soll. Es wurde ein Markdown erstellt.
Danach habe meine Tasks, beim bereits erstellen Gite Projekt hinzugefügt
und angepasst.

\vspace{0.5cm} \textbf{2022-08-17: 12.00 $\rightarrow$ 14.00}

Sp\"ater habe ich bei meiner Registrierungsseite weitergearbeitet
(Validation) und fertiggestellt. Danach fiel mir auf, dass der Dark bzw.
Light Theme sehr kompliziert von mir implementiert wurde. Ich entschied
mich, diesen zu ändern, um Komplikationen bei größerem
Projektfortschritt vermeiden zu können. Die restliche Zeit schaute ich
noch ein Video, was die beste Methode um dies zu programmieren

\vspace{0.5cm} \textbf{2022-08-17: 20:00 $\rightarrow$ 23.00}

Nach Recherche habe ich damit begonnen, einen möglichst schnelle und kurze
Variante zu bauen, um das Theme zu ändern. Nach 1,5 Stunden ging dieser
auch. Allerdings habe ich danach versucht verschiedene Werte in eine
lokale DB zu speichern, was sich leider als ein Problem darstellte.

\vspace{0.5cm} \textbf{2022-08-18: 12:30 $\rightarrow$ 16.30}

Nach etwas Recherche stellte sich mein Fehler heraus, den ich bei der
lokalen Datenbank gemacht habe. Danach habe ich etwas bei der
Ordnerstruktur umgeändert, um eine möglichst gute Performance zu
erreichen. Ebenfalls habe ich mit der Homepage angefangen und mir die
Dokus zu Menünavigation durchgelesen. Später hatte ich ein Meeting mit
Herrn Mayrhofer, um über den aktuellen Projektfortschritt zu reden.
Danach zeigte mir Herr Mayrhofer noch sein go Programm, um automatisch
die Arbeitsstunden zählen zu können.

\vspace{0.5cm} \textbf{2022-08-18 : 19:00 $\rightarrow$ 21:04}

Es wurde der Drawer für den Client Login erstellt, um auf verschiedene
Seiten zu wechseln. Dieser hat etwas Zeit gekostet, da ich diesen
möglich open-closed programmieren wollte, um Erweiterungen einfach
zu machen. Schlussendlich funktionierte dann alles. Danach wurden noch
verschiedene Designideen geändert.

\vspace{0.5cm} \textbf{2022-08-19: 9:20 $\rightarrow$ 11:20}

Am Vormittag hatte ich ein Meeting mit Herrn Zöchmann, um über den
aktuellen Projektfortschritt bzw. Aufteilung zu reden. Weiters haben wir
noch über eine Vereinheitlichung des Designs geredet und uns für ein
gleiches entschieden. Allerdings trat ein Problem mit der Helligkeit der
Farben auf, wenn man den Dark und Lightmode änderte, was etwas Zeit
raubte. Danach haben wir unsere RegEx geändert und etwas angepasst

\vspace{0.5cm} \textbf{2022-08-19: 11:45 $\rightarrow$ 14:08}

Sp\"ater habe ich begonnen die Seite für die Karten zu erstellen. Zum
Testen der Karten wurde eine publike API verwendet. Um eine möglichst
gute Erweiterbarkeit zu ermöglichen, versuchte ich ein Konzept zu
erstellen. Allerdings stellte sich die Umsetzung als Problem, sehr
sonderbar ist in Sachen OOP. Als Unterstützung half mir Herr Mayrhofer.
Allerdings hat er auch keine Lösung gefunden.

\vspace{0.5cm} \textbf{2022-08-19: 17:00 $\rightarrow$ 19:00}

Nach etwas Pause versuchte ich erneut einen allgemeinen Handler für Get
Api Anfrage zu schreiben und automatisch Widget davon in der App zu
erstellen. Nach etwas genauerer Überlegung fand ich das Problem und
schaffte den Handler erfolgreich zu erstellen.

\vspace{0.5cm} \textbf{2022-08-22: 9:00 $\rightarrow$ 12:08}

Heute habe ich bei der Kartenansicht weitergearbeitet. Es werden nun
alle Karten je nachdem ob sie reserviert sind oder derzeit benutzt
werden angezeigt. Allerdings hatte ich ein paar Probleme bekommen, da der
DateTimepicker nicht ordentlich funktionierte. Schlussendlich hat dann
alles funktioniert. Weiters habe ich mir aufgeschrieben, wie das
Reservierungssystem aufgebaut werden soll.

\vspace{0.5cm} \textbf{2022-08-22: 14:30 $\rightarrow$ 16:30}

Am Nachmittag habe ich beim Reservierungssystem weitergearbeitet. Nach
Fertigstellung eines Pop-ups (zum Auswählen der Reservierungszeit
(von, bis)) bekam ich eine Fehlermeldung, zu der ich keine Lösung fand.
Nach etwas Probieren konnte ich allerdings das Problem lösen. Danach
habe ich das Pop-up-Fenster zum NFC scannen geschrieben

\vspace{0.5cm} \textbf{2022-08-22: 21:00 $\rightarrow$ 22:13}

Abends versuchte ich dann die Kartenansicht, damit sie auch auf anderen
Geräten gleich aussieht. Allerdings stellte sich das als Problem dar,
da Flutter angesichts dieses Themas sehr komisch ist. Auch nach
zahlreichen Besuchen in Foren fand ich keine saubere Lösung, sondern nur
Workarounds. Morgen habe ich vor, eine saubere Implementation zu finden

\vspace{0.5cm} \textbf{2022-08-23: 09:50 $\rightarrow$ 12:50}

Am Vormittag habe ich versucht, die Karten-Seite mit Constraints auf allen
Ger\"aten gleich aufzulösen. Allerdings hat dies wieder nicht
funktioniert. Danach habe ich die komplette Karten-Seite neu strukturiert, da mir die Ordnerstruktur nicht mehr gefiel. Sp\"ater habe
ich mit der Reservierungsseite begonnen, die die reservierten Karten pro
Person anzeigt.

\vspace{0.5cm} \textbf{2022-08-23: 16:50 $\rightarrow$ 18:00}

Am Nachmittag hatte ich ein einstündiges Meeting mit Herrn Mayrhofer,
um die von ihm erstellte API zu testen. Anfangs hatten wir noch
Probleme, da wir den Localhost des Emulators der IP-Adresse vom
Computer verwechselt haben. Danach konnte ich erfolgreich Daten von der
API zu holen. Danach hatten wir noch ein Gespräch über die Aufteilung
und Funktionsweise des Projekts. Es wurde auch über die
Datenbankstruktur diskutiert

\vspace{0.5cm} \textbf{2022-08-23: 21:15 $\rightarrow$ 22:35}

Abends habe ich dann das Design der Kartenansicht völlig überarbeitet,
nun wird die Seite, egal wie groß das Display, gleich angezeigt.
Ebenfalls wurde der Login überarbeitet, damit er auch responsive ist

\vspace{0.5cm} \textbf{2022-08-24: 10:20 $\rightarrow$ 12:53}

Gegen Mittag habe ich die Reservierungsseite erstellt und beim Code, der
die Wildcards generiert dementsprechend etwas umgeändert, sodass er
schön erweiterbar ist. D.h je nach Seite (Reservierung oder Karten)
werden die Karten generiert und angepasst a. Beim Testen ist
mir dann ein Fehler bei der API aufgefallen, den ich dann Johannes
erklärt habe. Gegen Abend habe ich dann noch die ausgebesserte API von
Johannes erfolgreich testen können

\vspace{0.5cm} \textbf{2022-08-24: 14:10 $\rightarrow$ 15:50}

Am Nachmittag habe ich dann noch die Funktion hinzugefügt, um Daten
senden zu können. Allerdings hatte ich verschiedene Probleme, da das
Encodieren meiner Daten nicht Daten vollst\"andig funktionierte. Ebenfalls habe ich
mir die Flutter Dokumentation durchgelesen, damit ich besser weiß, wie man Projekte besser aufbaut.

\vspace{0.5cm} \textbf{2022-08-24: 19:40 $\rightarrow$ 21:15} 

Am Abend habe ich die Business-Logik für die Reservierungsseite
programmiert. Es können jetzt ganz einfach Reservierungen aufgehoben,
bearbeitet werden. Um das open-closed Prinzip einhalten zu können, habe
ich mir ebenfalls dafür einen guten Aufbau überlegt

\vspace{0.5cm} \textbf{2022-08-25: 9:00 $\rightarrow$ 11:30}

Am Vormittag hatte ich ein Meeting mit Ben. Am Anfang half ich ihm bei
seinem Issue, und löste ihn mit einer Callbackfunction (delegate).
Danach erklärte ich ihm unsere Ordnerstruktur. Weiters habe ich ihm
erklärt, wie mein API-Visualizer funktioniert. Danach zeigte ich ihm,
wie man die API von Herrn Mayrhofer startet und damit kommuniziert.
Danach brachte ich ihm bei, wie man die Zählstunden Anwendung von Herrn MJ
nutzt. Ebenfalls haben wir noch darüber geredet, wie unser git merge
ablaufen wird

\vspace{0.5cm} \textbf{2022-08-25: 15:00 $\rightarrow$ 18:00}

Am Nachmittag habe ich bei der Reservierungen - und Kartenseite
weitergearbeitet. Ich habe nun für das Reservierungssystem den Put API
Call programmiert. Danach habe ich eine neue Seite erstellt, um die
Benutzerdaten zu ändern. Allerdings hat mir ein Problem mit Flutter
(Ordnernamen umbenennen) etwas Zeit geraubt, da dies nicht richtig
funktioniert und Flutter die neuen Namen nicht finden konnte

\vspace{0.5cm} \textbf{2022-08-25: 20:30 $\rightarrow$ 00:02}

Am Abend habe ich aufgrund eines Gesprächs mit Herrn Zöchmann meinen
Drawer völlig entfernt und Tabs eingebaut. Dazu überlegte ich mir
wieder ein Konzept, damit es schön erweiterbar ist. Danach habe ich
eine Seite Settings erstellt und dort verschiedene Verlinkungen
eingebunden. Außerdem habe ich das komplette Colorscheme der App
verändert, was etwas Zeit kostete.

\vspace{0.5cm} \textbf{2022-08-26: 8:45 $\rightarrow$ 10:45}

Am Vormittag hatte ich ein Meeting mit Herrn Zöchmann bezüglich des
Designs der App. Ebenfalls implementierten wir eine Standardfont.
Danach habe ich bei der Settings-Seite etwas weitergearbeitet

\vspace{0.5cm} \textbf{2022-08-26: 13:30 $\rightarrow$ 16:07}

Am Nachmittag wollte ich beim Reservierungs-Pop-Up die Validation
hinzufügen. Dies stellte sich allerdings als Problem dar, da ein Pop-up
kein Stateful Widget ist. Deshalb überlegte ich mir ein Konzept, um
dem automatisch generierten Pop-up auch eine Überprüfung zu geben.
Dies benötigte etwas Zeit. Als ich fertig war, übte ich dann noch
kleine Verbesserungen bei der App aus.

\vspace{0.5cm} \textbf{2022-08-29: 9:00 $\rightarrow$ 12:15}

Am Vormittag fügte und änderte ich verschiedene Tasks im Git Project.
Danach entschied ich mich, die eine, die Benachrichtigungsservice zu schreiben,
der sowohl auf iOS als auch Android funktioniert, wenn die App
geschlossen ist. Das Problem war, da ich kein Paket dafür finden
konnte. Einige Zeit später fand ich ein Tutorial, dass ich nachmachen
wollte, aber dann nicht funktioniert. Schlussendlich entschied ich
mich, eine fertige Version von GitHub runterzuladen, und diese an mein
Projekt dann anzupassen

\vspace{0.5cm} \textbf{2022-08-29: 13:00 $\rightarrow$ 15:11 Uhr}

Nach dem Mittagessen implementierte ich den Notifikationservice in
meine App. Zunächst funktionierte es, allerdings trat dann ein Problem
auf, da die Reservierungszeiten, einen anderen Timestamp verwenden als
der Service. Um dies zu begreifen, benötigte es etwas Zeit.
Schlussendlich konnte ich das Problem finden und somit erfolgreich das
Reservierungssystem vervollständigen

\vspace{0.5cm} \textbf{2022-08-30: 9:00 $\rightarrow$ 12:00}

Am Vormittag habe ich mir als Ziel genommen einen E-Mail Bot zu schreiben,
der die Best\"atigung der Anmeldung und das Zurücksetzen des Passworts
übernimmt. Dazu habe ich mir ein Tutorial angeschaut und
dementsprechend im Code angepasst. Zwischendurch hatte ich noch ein
Meeting mit Herrn Zöchmann, um uns über den aktuellen Projektstatus
auszutauschen. Ebenfalls haben wir gemeinsam bei mir ein Error beim
Kompilieren beseitigt, da ein verwendetes Paket einen Fehler beim Code
hatte.

\vspace{0.5cm} \textbf{2022-08-30: 12:30 $\rightarrow$ 14:00}

Nach dem Mittagessen habe ich ein neues Scaffold zum Zurücksetzten des
Passworts erstellt, welches den Informationsaustausch zwischen User und
UI ermöglicht. Zwischendurch, kam ich dann auf die Idee das es gut wäre,
die Struktur an das DRY Prinzip anzupassen. Allerdings, habe ich mich nur
informiert, wie dieses zum Umsetzen ist und habe mir ein Konzept
überlegt, wie ich dieses einbauen möchte

\vspace{0.5cm} \textbf{2022-08-30: 15:30 $\rightarrow$ 18:10}

Gegen Abend hin habe ich dann mein Konzept, was ich mir überlegt habe, um
das DRY Prinzip zu implementieren umgesetzt. Und getestet. Danach habe
ich beim Login noch etwas geändert. Ebenfalls wurde eine Vorlage erstellt, welches
überprüft, ob die Login-Daten übereinstimmen.

\vspace{0.5cm} \textbf{2022-08-31: 10:00 $\rightarrow$ 12:00}

Heute früh hatte ich ein Meeting mit Herrn Zöchmann, um über
die Konstruktionskonzepte des anderen Teams zu reden. Danach habe ich
Zöchmann mein Konzept gezeigt, um das DRY Prinzip umzusetzen. Danach
habe ich noch ein paar Bugfixes bei der App ausgeführt

\vspace{0.5cm} \textbf{2022-08-31: 13:00 $\rightarrow$ 16:00}

Am Nachmittag habe beim E-Mail Bot (Passwort Reset) weitergeschrieben.
Als der Bot erfolgreich die Mails und den User schicken konnte, wollte
ich einen Link erstellen, der in der Mail ist, um die App zu \"offen und
ein neues Passwort einfügen zu können. Allerdings stellte sich dies
als ein Problem dar, da Flutter keinen Support f\"ur Deeplinks aufweist.
Nach sehr, sehr langer Suche fand ich allerdings ein Paket, welches ein
einem Forum empfohlen wurde. Derzeit bin ich noch dabei, dieses zu
implementieren, was sich als große Herausforderung herausstellte.

\vspace{0.5cm} \textbf{2022-08-31: 18:30 $\rightarrow$ 19:30}

Am Abend versuchte dann Herr Zöchmann mir dann noch mit den Deep links
zu helfen. Allerdings schafften wir es zu weit auch nicht.

\vspace{0.5cm} \textbf{2022-09-01: 9:30 $\rightarrow$ 11:35}

Am Vormittag habe ich die Suchleiste bei den Karten eingefügt. Beim
Einbauen ist mir aufgefallen, dass ich bei der Struktur etwas
vereinfachen kann. Nachdem ich es vereinfacht habe, wurde die Suchleiste
hinzugefügt und getestet. Danach wurde noch etwas bei der Account Seite
umgeändert, um sie performanter zu machen

\vspace{0.5cm} \textbf{2022-09-01: 13:00 $\rightarrow$ 15:00}

Am Nachmittag habe ich mich damit besch\"aftigt, die Ordner und Filename der
Konvention anzupassen. Dies stellte sich als Herausforderung dar, da ich nicht wusste, dass Flutter die Ordner bzw. Filename in einen eigenen Ordner speichert. Nach zahlreichen Versuchen suchte ich mir die Lösung in einem Forum raus.

\vspace{0.5cm} \textbf{2022-09-02: 9:00 $\rightarrow$ 12:40}

Am Vormittag habe ich bei der App, dass DRY Prinzip fuer die
Buttons geschrieben, um Zeilen sparen und eine bessere Lesbarkeit zu
ermöglichen. Danach habe ich einen Intent geschrieben, um die
Einstellungen vom Betriebssystem öffnen zu können, damit öffnen
Benachrichtigungseinstellungen ge\"andert werden können. Sp\"ater habe ich noch ein
paar Bugs behoben. Allerdings war die App soweit fertig, aber ich musste
noch auf die Fertigstellung von MJsAPIApi warten. Deshalb habe ich
begonnen, den NFC Writer beim Handy zu implementieren, was sich als
Herausforderung darstellt, da es keine Pakete dafür gibt.

\vspace{0.5cm} \textbf{2022-09-06: 8:00 $\rightarrow$ 11:15}

Am Vormittag hatte ich ein Meeting mit Herrn Zöchmann, um über den
aktuellen Projektstand zu reden. Danach habe ich das Programm für den
NFC Reader am Raspberry geschrieben. Nach langem Suchen in zahlreichen
Foren, stellt sich raus, dass mein Konzept mit dem NFC Lesen nicht
möglich ist, da sich die UID laufend änderte. Danach habe ich mir ein
neues Konzept überlegt und hatte anschließend wieder ein Meeting mit
Herrn Zöchmann.

\vspace{0.5cm} \textbf{2022-09-06: 13:00 $\rightarrow$ 15:00}

Am Nachmittag habe ich mich damit beschäftige, einen Filter
hinzuzufügen. Zuerst überlegte ich mir ein Konzept, um eine möglichst
gute Implementierung zu haben. Danach wurde der Filter erfolgreich
einprogrammiert. Allerdings muss ich noch auf die Fertigstellung von MJ
seiner API warten und den Filter fertigzustellen.

\vspace{0.5cm} \textbf{2022-09-06: 21:00 $\rightarrow$ 23:30}

Abends habe ich dann noch versucht, den Filter fertigzustellen.
Allerdings stieß ich auf ein Problem, da ich leider bei meinem Konzept
die falsche Liste verwendet habe. Nach etwas Debuggen konnte ich
allerdings den Fehler finden und behebe. Danach wurde noch etwas am
Design umge\"andert

\vspace{0.5cm} \textbf{2022-09-08: 8:00 $\rightarrow$ 9:10}

Am Vormittag hatte ich ein Meeting mit Herrn Zöchmann, da er Hilfe beim
Refactoring benötigt hat. Ebenfalls wurden zahlreiche organisatorische
Sachen bzgl. der Diplomarbeit besprochen.

\vspace{0.5cm} \textbf{2022-09-08: 13:30 $\rightarrow$ 15:30}

Am Nachmittag habe ich mit dem Errorhandling begonnen. Ich durchsuchte
zahlreiche Foren, um die beste Methode zu finden. Allerdings hatte ich ein
Problem, um den Futurebuilder ein Errorhandling zu machen. nach
etwas Probieren schaffte ich es dann. Danach habe ich es Herrn Zöchmann
gezeigt und um Übereinstimmung gebeten. Ebenfalls habe ich noch mein
Git Project aktualisiert, da ich darauf vergessen habe

\vspace{0.5cm} \textbf{2022-09-16: 08:15 $\rightarrow$ 12:30}

Am Freitag in der ITP Stunde mussten wir auf der Diplomarbeitsdatenbank
einen Projektantrag erstellen. Danach hatten wir eine Besprechungsstunde
mit Frau Hofer bzgl. der Diplomarbeit. Es wurden viel Sachen
aufgeklärt. Danach habe ich eine kurze Zusammenfassung für ein
Pflichtenheft geschrieben und eine Artikelliste erstellt. Danach wollte
ich bei der App weiterarbeiten, allerdings hatte ich zahlreiche kuriose
Probleme bei der Installation. Die Vm wollte nicht laufen.

\vspace{0.5cm} \textbf{2022-09-17: 8:00 $\rightarrow$ 11:30} 

Heute hatten wir ein
Teammeeting, da wir unser Konzept überarbeiten wollten und ein
Pflichtenheft erstellen wollten. Es wurde alles erledigt

\vspace{0.5cm} \textbf{2022-09-19: 10:30 $\rightarrow$ 14:30} 

Flutter wurde am Laptop installiert.Allerdings hatte ich zahlreiche Probleme bei der Installation. Die IDE wurde mehrmals neu
heruntergeladen, allerdings hat das nicht geholfen. Auch nach dem Besuch
von zahlreichen Foren konnte ich keine Lösung finden, die VM zu
starten. Zu Hause habe ich mich dann entschieden eine ältere Version von
Android Studio zu verwenden, was dann schlussendlich das Problem löste.
Danach habe ich noch verschiedene Buttons neu erstellen müssen, da diese
in der neuen Flutter Version nicht mehr verfügbar sind

\vspace{0.5cm} \textbf{2022-09-22: 11:30 $\rightarrow$ 14:30} 

Heute habe ich bei der
App neue Validation hinzugefügt. Danach habe ich die Passwort create
Option in ein eigenes File ausgelagert, um es generisch generieren zu
lassen. Danach habe ich mir eine neue Logik überlegt ohne Intentions das
Passwort zurück zusetzten. Lösung, das Versenden einer Mail. Ebenfalls wurde eine Klasse erzeugt, die diesen Code
automatisch erzeugt

\vspace{0.5cm} \textbf{2022-09-28: 15:30 $\rightarrow$ 20:30} 

Habe heute die Hauptseite erstellt. Diese wird als Favoritenseite für die Karten
benutzt. Das Ganze wurde so erstellt, dass es keine Probleme darstellen soll, wenn man den Karten Reminder implementieren soll. Danach habe ich im Internet nach einer Lösung gesucht, den Login via Microsoft zu erledigen. Nach zahlreichen gescheiterten Versuchen kam ich zu dem Entschluss Firebase zu verwenden. Was sich als gute Idee herausstelle, da ich Firebase auch für die Benachrichtigungen, sofern sich ein Wert bei der API änderte

\vspace{0.5cm} \textbf{2022-09-29: 11:15 $\rightarrow$ 14:15} 

Habe heute bei der Favoriten-Seite weitergearbeitet. Es wurde eine Klasse
erstellt, die die automatische Generierung von Buttons für eine Karte erm\"oglicht. Danach wurden noch ein paar Bugs gefixt und nach der Firebase impl. gesucht

\vspace{0.5cm} \textbf{2022-09-30: 8:15 $\rightarrow$ 11:55} 

In dieser Stunde hatte ich eine  Diskussion mit meinen Kollegen aufgrund der
Authentifizierung mittel Microsoft. Ebenfalls hatten wir ein Meeting
aufgrund dessen mit Frau Hofer. Danach wurde bei der App noch etwas noch
am Design rändert und eine Benachrichtigungsfunktion eingebaut, falls die Karte
wieder frei ist.

\vspace{0.5cm} \textbf{2022-09-30: 16:00 $\rightarrow$ 18:00} 

Das UC wurde überarbeitet

\vspace{0.5cm} \textbf{2022-09-30: 16:00 $\rightarrow$ 18:00} 
Das UC wurde überarbeitet

\vspace{0.5cm} \textbf{2022-10-05: 12:00 $\rightarrow$ 16:00}


Es wurde eine Seite erstellt, um Mails an andere Benutzer zu senden, die gerade eine Karte verwenden, die sie benötigen. Allerdings bemerkte ich dann, dass diese Implementation falsch ist und erstellte eine Intention, die automatisch eine Liste der
installierten Mail-Apps anzeigt und sofern ein Mail App ausge\"ahlt wurde eine Vorlage der Mail erstellt

\textbf{2022-10-08: 09:00 $\rightarrow$ 11:30}

Heute wurde ein Probebeispiel zu dem Microsoft Login erstellt. Dazu verwendete ich die Doku, die mir Hasp schickte. Ich las mir diese durch und erstellte ein Projekt,
welches als Beispiel diente. Es funktionierte alles perfekt

\vspace{0.5cm} \textbf{2022-10-08: 08:15 $\rightarrow$ 10:00} 

Heute habe ich Ben geholfen, eine Benachrichtigung mittels API zu erstellen. Danach habe
ich den Login und .env bei der RFID App hinzugefügt

\vspace{0.5cm} \textbf{2022-10-27: 10:00 $\rightarrow$ 13:00} 

Wie mit Herrn Haslinger besprochen, habe ich mir die Doku von Microsoft Graph durchgelesen, um an die User Daten zu kommen. Danach verwendeten ich den Graph Explorer, um
die Daten von meinem eigenen Access Token zu bekommen, was auch
funktionierte. Danach wollte ich mit einer API diese Daten holen, wo ich
allerdings auf ein Problem trat. Ich hatte keine Rechte, mich anzumelden.
Nach zahlreichen Foren und einer Mail ans Herrn Haslinger funktionierte
es immer noch nicht. Am Nachmittag habe ich es dann nochmals
probiert und lustigerweise hat es dann funktioniert

\vspace{0.5cm} \textbf{2022-10-29: 22:00 $\rightarrow$ 24:00} 

Heute habe ich die JSON Antwort, die bei der API zurückgegeben wird, geparst. Dazu habe ich einen bereits von mir erstellten Parser verwendeten (Cards). Allerdings
hat dies dann nicht auf Anhieb funktioniert. Nach langer Zeit und
Debuggen habe ich mitbekommen, dass ich in meiner JSON einen
Rechtschreibfehler hatte. Danach hat alles funktioniert

\vspace{0.5cm} \textbf{2022-10-30: 21:00 $\rightarrow$ 22:15} 

Es wurden Änderung am User Type (JSON Klasse) gemacht und die Account Seite überarbeitet, der den eingeloggten User anzeigt

\vspace{0.5cm} \textbf{2022-11-12: 10:00 $\rightarrow$ 14:25} 

Heute habe ich die Funktion zum Speichern des Logins, f\"ur Microsoft implementiert.
Allerdings konnte ich im initState mich nicht automatisch anmelden,
deshalb suchte ich im Internet nach einer Lösung. Nach viel Versuchen schaffte ich es allerdings nicht und löste das Problem mit einem Workaround. Nun funktioniert alles bis auf die Firebase API Benachrichtigung

\vspace{0.5cm} \textbf{2022-11-12: 10:00 $\rightarrow$ 12:00} 

Es wurde eine Login-Seite für den Admin und User erstellt. Der User Login wurde
überarbeitet. Weiters wurde für das E-Mail Pop-up eine generische Klasse
geschrieben

\vspace{0.5cm} \textbf{2022-11-18: 15:00 $\rightarrow$ 16:00} 

Es wurde für den RFID Scanner die Halterung getestet.

\vspace{0.5cm} \textbf{2022-11-19: 09:00 $\rightarrow$ 13:00} 

Es wurde für das
Display die App erstellt und angepasst. Es wurde ebenfalls versuchte ich eine
App auf Windows und nicht über den Browser zu debuggen, allerdings ohne
Erfolg. Bei der Card Visualisierung im Browser kann ich aus irgendwelchen
Gründen nicht auf den Localhost zugreifen. Es wurde dazu eine Frage auf
Stackoverlow gestellt. Danach half ich Herrn Zöchmann bei seiner App.

\vspace{0.5cm} \textbf{2022-11-19: 10:00 $\rightarrow$ 17:15} 

Es wurde für die Display-App eine Pop-up erstellt welches einen Timer visualisiert zur Anzeige der restlichen Zeit.

\vspace{0.5cm} \textbf{2022-11-21: 9:50 $\rightarrow$ 13:50} 

Tutorial zu Threading wurde angesehen. Und ein Beispielprogramm wurde erzeugt. Danach wurde der Thread im Hauptprogramm implementiert und es wurde ein freier Code block für
das MQTT Protokoll gefertigt

\vspace{0.5cm} \textbf{2022-11-22: 10:00 $\rightarrow$ 13:00} 

MQTT Protokoll wurde implementiert und angepasst. Weiters half ich Zöchmann bei einer
Problemstellung

\vspace{0.5cm} \textbf{2022-11-24: 12:00 $\rightarrow$ 14:00} 

Es wurde zum Testen des MQTT Protokolls ein MQTT Docker erstellt

\vspace{0.5cm} \textbf{2022-11-26: 09:30 $\rightarrow$ 13:40} 

Es wurde ein Raspberry aufgesetzt, der die Flutter Anwendung kompiliert und anzeigen soll. Ich verwendete dazu Raspbian OS Lite, um Ressourcen zu sparen. Im n\"achsten Schritt wurde Flutter-Pi installiert und konfiguriert

\vspace{0.5cm} \textbf{2022-11-26: 16:30 $\rightarrow$ 17:40} 

Properties File und JSON im MQTT wurde hinzugefügt, weiters wurde der Bildschirm nun um 90 Grad gedreht. Ebenfalls ist es jetzt möglich den MqttTimer zu stoppen

\vspace{0.5cm} \textbf{2022-11-27: 18:00 $\rightarrow$ 19:30} 

Telefongespräch mit MJ über MQTT und API. Ebenfalls wurde das Raspberry Programm zum
Kommunizieren mit der API, MQTT und Maschinenbauer erstellt

\vspace{0.5cm} \textbf{2022-11-24: 13:00 $\rightarrow$ 18:00} 

App wurde an API von
Johannes umgeändert und Refactoring wurde betrieben.

\vspace{0.5cm} \textbf{2022-12-07: 10:00 $\rightarrow$ 13:00} 

Es wurde ein Dokument
für die API bzw. MQTT erstellt. Dort wurde der Datenverkehr von App bis
zu Raspberry genausten beschrieben. Weiters hatten wir ein Meeting mit
den Maschinenbauern

\vspace{0.5cm} \textbf{2022-12-07: 18:00 $\rightarrow$ 21:45} 

Beide Programme wurden refactort und der neuen Struktur von MJ angepasst.

\vspace{0.5cm} \textbf{2022-12-08: 20:00 $\rightarrow$ 23:30} 

Es wurde beim Login das
Setup zum Registrieren erstellt: Neues Pop-up, neuer Timer, vordefinierte Zeilen, falls die API fertig ist und an Johannes seiner API angepasst

\vspace{0.5cm} \textbf{2022-12-26: 17:00 $\rightarrow$ 20:30} 

Es wurde das Python
Programm vom Terminal für den Sensor programmiert. Dies war zun\"achst
etwas schwierig, da gewissen Voraussetzungen noch nicht fertig
programmiert wurden.

\vspace{0.5cm} \textbf{2023-01-05: 20:00 $\rightarrow$ 23:10} 

Nach einwöchiger Pause aufgrund einer Grippe ging es heute wieder los. -Besprechung mit Herrn Mayrhofer über den aktuellen Status -Start der Implementation der neuen
API Version

\vspace{0.5cm} \textbf{2023-01-06: 09:00 $\rightarrow$ 14:00} 

Implementation der neuen API Struktur von Johannes. Vollständige Codebase überarbeitung inDisplayapp.

\vspace{0.5cm} \textbf{2023-01-06: 17:00 $\rightarrow$ 20:00} 

Start neuer App für Websocket. Lesen von online Doku

\vspace{0.5cm} \textbf{2023-01-07: 09:00 $\rightarrow$ 13:00} 

Implementation neue Api Struktur in zweite App, Vollständige Codebase refactoring in RfidApp, Implementation von Websocket

\vspace{0.5cm} \textbf{2023-01-07: 17:00 $\rightarrow$ 20:00} 

Websocket Testing und Codeanpassung. Ebenfalls wurde Simulator von Johannes umgeschrieben. Leider stieß ich auf ein Problem, da MJ noch keine Websocket f\"ur mehrere Benutzer erstellt hat.

\vspace{0.5cm} \textbf{2023-01-08: 10:00 $\rightarrow$ 12:00} 

Aufgrund eines Problems mit dem Circular Timer wurde das Paket heruntergeladen und erweitert

\vspace{0.5cm} \textbf{2023-01-10: 10:00 $\rightarrow$ 11:00} 

Umprogrammierung des Websocket, das pro Start der App eine Verbindung gestartet wird und nicht durchgehend

\vspace{0.5cm} \textbf{2023-01-11: 8:15 $\rightarrow$ 12:00} 

Umprogrammierung des Websocket, dass pro Start der App eine Verbindung gestartet wird und nicht durchgehen und umprogrammieren in ein responsives Programm

\vspace{0.5cm} \textbf{2023-01-13: 13:00 $\rightarrow$ 17:00} 

Tag der offenen Tür

\vspace{0.5cm} \textbf{2023-01-14: 08:15 $\rightarrow$ 13:00} 

Tag der offenen Tür

\vspace{0.5cm} \textbf{2023-01-21: 14:00 $\rightarrow$ 18:00} 

Implementation des Reservierungssystems

\vspace{0.5cm} \textbf{2023-01-22: 20:00 $\rightarrow$ 21:40} 

Es wurden Bugs bei den Reservierungen behoben. Die RfidApp ist soweit fertig.

\vspace{0.5cm} \textbf{2023-01-31: 15:00 $\rightarrow$ 19:00} 

Heute wurde der Request Timer bei der Display-App erstellt, falls nun  eine Mqttnachricht mit der action user-sign-up gsendet, d.h wenn ein User sich registrieren
möchte, erscheint nun ein Pop-up am Display. Allerdings hatte ich ziemliche Schwierigkeiten, und musste von der Logik viel ändern.
Schlussendlich musste ich sogar an der zweiten App noch etwas Logik bei
der Websocket Implementierung ändern

\vspace{0.5cm} \textbf{2023-02-01: 18:00 $\rightarrow$ 20:00} 

Heute musste ich ein
paar Fehler beheben. Zunächst wurde etwas beim Login geändert. Danach
l\"oste ich die Probleme, die mir VsCode anzeigte.

\vspace{0.5cm} \textbf{2023-02-07: 18:00 $\rightarrow$ 22:00} 

Es wurde der neue
Websocket von Herrn Mayrhofer bei der Display-App implementiert. Da Herr
Mayrhofer die Authentifizierung hinzufügte, musste ich mir ein Konzept
überlegen, um nicht bei jeder Methode den Zustand des Tokens zu
überprüfen. Lösung war eine Middleware, welche die Funktion für den API
Aufruf und Argumente als Parameter hatte.

\vspace{0.5cm} \textbf{2023-02-07: 19:00 $\rightarrow$ 00:00} 

Heute wurde der Login
vollständig überarbeitet. Es wurde eine Klasse Session User erstellt.
Diese Klasse bekommt bei jeder Anmeldung die entsprechenden Daten
übergeben. Aufgrund eines Videos auf YouTube überarbeitete ich ebenfalls
die Struktur des Projekts. Es wurden für Kinder-Widgets, Inheritated
Widgets verwendet, welche sozusagen als Middleware zwischen Eltern und
Kinder Widgets dient

\vspace{0.5cm} \textbf{2023-02-09: 21:00 $\rightarrow$ 23:00} 

Heute wurde eine Klasse
SnackbarBuilder erstellt, welches dazu dient ein Pop-Up bei Fehler, Events
anzuzeigen.

\vspace{0.5cm} \textbf{2023-02-10: 16:00 $\rightarrow$ 21:30} 

Heute habe ich mit Herrn
Zöchmann unsere Programme gemerged. Es dauert etwas, bis alles
funktionierte. Danach musste ich ein paar Sachen anpassen bzw. beheben.
Ebenfalls machte ich die ganze App responsive. Es wurde auch das
Errorhandling überarbeitet. Die App ist nun vollständig einsatzfähig.

\vspace{0.5cm} \textbf{2023-02-11: 18:00 $\rightarrow$ 00:00} 

Es wurde an der Dokumentation weiter gearbeitet, wobei ich bei mit der Implementierung
begann. Beim Erklären ist mir aufgefallen, dass gewisse Code Teile
verbessert werden konnte. Es wurde die Logik des responsiven Designs überarbeitet 

\vspace{0.5cm} \textbf{2023-02-12: 11:00 $\rightarrow$ 15:00} 

Weiterarbeiten an der Dokumentation (Implementierung). Verbesserung von Code Teilen (Auslagerungen verschiedener Teile, Klassenname wurden umbenannt) 

\vspace{0.5cm} \textbf{2023-02-12: 19:00 $\rightarrow$ 00:00} 

Weiterarbeiten an der
Dokumentation (Implementierung). Umsetzung verschiedener Design Patterns im Code

\vspace{0.5cm} \textbf{2023-02-13: 13:00 $\rightarrow$ 17:00} 

Es wurde an der Dokumentation (Architektur) weitergearbeitet. Ebenfalls wurde verschiedene Code Verbesserung an der Anwendung durchgef\"uhrt und verschiedene Bilder f\"ur die Dokumentation erstellt. Danach habe ich mit der Erkl\"arung der Implementierung begonnen. Geschrieben wurde über:  Responsive Design, Theme Handler.

\vspace{0.5cm} \textbf{2023-02-13: 21:00 $\rightarrow$ 23:00} 

Weiterarbeiten an der Dokumentation (Implementierung). Verbesserung von Code Teilen. Geschrieben wurde über: Theme Handler, Inherited Widget.

\vspace{0.5cm} \textbf{2023-02-14: 21:20 $\rightarrow$ 22:50} 

Weiterarbeiten an der Dokumentation (Implementierung). Verbesserung von Code Teilen. Geschrieben wurde über: API Verbindungen, Feedback Dialog

\vspace{0.5cm} \textbf{2023-02-15: 21:45 $\rightarrow$ 23:30} 

Weiterarbeiten an der Dokumentation (Implementierung). Verbesserung von Code Teilen. Geschrieben wurde über: Inherited Widget.Feedback Dialog, Darstellung von API Daten

\vspace{0.5cm} \textbf{2023-02-16: 10:00 $\rightarrow$ 17:15} 

Die Beschreibung der Implementierung in der Dokumentation wurde fertiggestellt. Weiters habe ich mit dem Thema Weiterentwicklung in der Dokumentation begonnen. Geschrieben wurde über: Anmeldeablauf, Weiterentwicklung (Installation Raspberry Pi und Erstellung einer APK) 

\vspace{0.5cm} \textbf{2023-02-16: 20:55 $\rightarrow$ 23:45} 

Das Thema Weiterentwicklung in der Dokumentation wurde fertiggestellt

\vspace{0.5cm} \textbf{2023-02-17: 8:15 $\rightarrow$ 19:30} 

Theorie wurde vollständig überarbeitet. Ebenfalls wurden Teile der Implementierung
geändert bzw. hinzugefügt. Es wurde das Benutzerhandbuch erstellt. Geschrieben wurde über: Responsive Design, Hot Reload, Asynchrone Programmierung, Modifier, Namenskonventionen, Umwandlung der API Daten, Benutzerhandbuch, Implementierung, Aufgaben der Sichten


\vspace{0.5cm} \textbf{2023-02-18: 10:15 $\rightarrow$ 15:30} 

Das Dokument wurde auf Rechtschreibfehler und Grammatikfehler überpr\"uft. Ebenfalls wurde das Protokoll hinzugef\"ugt und dementsprechend auf Fehler gepr\"uft.

\vspace{0.5cm} \textbf{2023-02-19: 12:15 $\rightarrow$ 15:20} 

Es wurden neue Grafiken f\"ur verschiedene Teile der Dokumentation gefertigt

\vspace{0.5cm} \textbf{2023-02-20: 12:15 $\rightarrow$ 15:30} 

Teile der Dokumentation wurden auf Rechtschreibfehler und Grammatikfehler korrigiert

\vspace{0.5cm} \textbf{2023-02-20: 16:15 $\rightarrow$ 22:00} 

Beide Dokumentationen wurden zusammengef\"uhrt



\vspace{0.5cm} \textbf{2023-02-23: 12:00 $\rightarrow$ 16:00} 
Verbesserung des Dokuments

\vspace{0.5cm} \textbf{2023-02-24: 18:00 $\rightarrow$ 21:00} 
Verbesserung des Dokuments

\vspace{0.5cm} \textbf{2023-02-25: 12:00 $\rightarrow$ 16:00} 
Verbesserung des Dokuments und Erstellung von Use-Case Diagramme

\vspace{0.5cm} \textbf{2023-02-26: 16:00 $\rightarrow$ 17:00} 
Verbesserung des Dokuments

\vspace{0.5cm} \textbf{2023-03-10: 12:00 $\rightarrow$ 21:00} 
Doku überarbeitet, Folder, Video und PowerPoint fertiggestellt.