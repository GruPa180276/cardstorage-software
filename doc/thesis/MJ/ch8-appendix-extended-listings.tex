\section[GORM Demonstration anhand eines konkreten Beispiels]{\large GORM Demonstration anhand eines konkreten Beispiels}
\vspace{-\baselineskip}
\begin{lstlisting}[style=goRaw,caption={\centering Ausführlicher Programmcode zu Listing \ref{lst:tech:gorm:ex1} (\nameref{lst:tech:gorm:ex1})}]
package main

import (
    "fmt"

    "gorm.io/driver/mysql"
    "gorm.io/gorm"
)

type Office struct {
    OfficeID   uint     `gorm:"primaryKey"`
    Department string   `gorm:"default:'IT'"`
    Workers    []Worker `gorm:"foreignKey:SSN"`
}

type Worker struct {
    SSN       uint   `gorm:"primaryKey"`
    FirstName string `gorm:"type:varchar(32)"`
    LastName  string `gorm:"type:varchar(32)"`
    OfficeID  uint   `gorm:"foreignKey:OfficeID"`
    Office    Office
}

const (
    user     = "root"
    password = "root"
    hostname = "127.0.0.1"
    port     = 3306
    database = "offices"
)

func main() {
    connstring := fmt.Sprintf("%s:%s@tcp(%s:%d)/%s", 
        user, password, hostname, port, database)
      
    var db *gorm.DB
    var err error

    db, err = gorm.Open(mysql.Open(connstring))
    
    if err != nil {
        panic(err)
    }
    
    db.AutoMigrate(&Worker{}, &Office{})
}
\end{lstlisting}

\newpage
\section{\nameref{lst:tech:go:enum:ex1}}
\begin{lstlisting}[style=goRaw,caption={\centering Ausführlicher Programmcode zu Listing \ref{lst:tech:go:enum:ex1} (\nameref{lst:tech:go:enum:ex1})}]
package main

import "fmt"

const (
    Jeep = iota + 1
    _
    Audi
    Mercedes
    Volkswagen
    LandRover
)

func main() {
    fmt.Println(Jeep, Audi, Mercedes, Volkswagen, LandRover) 
}

// Ausgabe: 1 3 4 5 6
\end{lstlisting}

\newpage
\section{\nameref{lst:go:sync:concurrency:ex4}}
\begin{lstlisting}[style=goRaw,caption={\centering Ausführlicher Programmcode zu Listing \ref{lst:go:sync:concurrency:ex4} (\nameref{lst:go:sync:concurrency:ex4})}]
package main

import (
    "fmt"
    "math/rand"
    "sync"
    "time"
)

var lock *sync.Mutex = &sync.Mutex{}

func update(numberPtr *int) {
    for {
        lock.Lock()
        *numberPtr = rand.Intn(1000)
        lock.Unlock()
    }
}

func main() {
    var number int
    
    go update(&number)
    go update(&number)
    
    for {
        lock.Lock()
        fmt.Println(number)
        lock.Unlock()
        time.Sleep(250 * time.Millisecond)
    }
}
\end{lstlisting}

\newpage
\section{\nameref{lst:go:sync:concurrency:ex5}}
\begin{lstlisting}[style=goRaw,caption={\centering Ausführlicher Programmcode zu Listing \ref{lst:go:sync:concurrency:ex5} (\nameref{lst:go:sync:concurrency:ex5})}]
package main

import (
    "fmt"
    "math/rand"
    "sync"
)

func greet(person string) string {
    return fmt.Sprintf("%s %s!", []string{"Hello,", "Howdy", "Greetings,", "Hi"}[rand.Intn(4)], person)
}

func main() {
    var wg sync.WaitGroup
    var cond *sync.Cond = sync.NewCond(&sync.Mutex{})
    
    people := []string{"Nick", "Alecia", "Madison", "Victor", "Travis"}
    greetings := []string{}
    
    wg.Add(1)
    go func() {
        cond.L.Lock()
        for len(greetings) != len(people) {
            cond.Wait()
        }
        for _, greeting := range greetings {
            fmt.Println(greeting)
        }
        cond.L.Unlock()
        wg.Done()
    }()
    
    for _, person := range people {
        cond.L.Lock()
        greetings = append(greetings, greet(person))
        cond.L.Unlock()
    }
    
    cond.Broadcast()
    
    wg.Wait()
}
\end{lstlisting}