\textbf{4.7.2022: 15:00 $\rightarrow$ 19:30} \

Generelle Recherche für Diplomarbeit. Unter anderem für mögliche Displays. Weiters wurde noch zur Programmierung der Anwendung für das Display recherchiert. Besprechung mit Teamkollegen (2h 30min Teams Call). Wir haben uns drauf geeinigt, 2 verschiedene Anwendungen zu erstellen, eine Client und eine Admin Anwendung. Diese werden beide mit Flutter realisiert und sind sowohl am Smartphone, Browser und am Raspberry am Kartentresor verfügbar. Weiters wurden sich Gedanken über die Funktionen gemacht, welche implementiert werden sollen. Zudem wurde ein weiter Teams Call durchgeführt. (4h 30min) 

\vspace{0.5cm}

\textbf{5.7.2022: 14:45 $\rightarrow$ 15:45} \

Teams Call mit Patrick über das Design der App. Es wurden das Design, sowohl auch das Farbschema abgestimmt und es wurden sich Gedanken über das Login Layout gemacht. (1h)

\vspace{0.5cm}

\textbf{6.7.2022: 15:00 $\rightarrow$ 16:00} \

Allgemeine Informationen über die Admin App zusammengeschrieben. Weiters wurden 2 weitere Branches erstellt und sämtliche Readmes mit Informationen befüllt. (1h)

\vspace{0.5cm}

\textbf{7.7.2022: 18:00 $\rightarrow$ 22:00} \

Erstes Dokument mit Latex erstellt. Hilfreiche Links \href{https://www.overleaf.com/learn/latex/Line_breaks_and_blank_spaces#Page_breaks}{Doku}. Weiters wurde ein
Übungsdokument erstellt, \href{https://www.overleaf.com/project/62c5bdaad31d926d54df41dd}{hier}. Weiters wurde ein gesamtes Konzept erstellt. Unser Konzept haben wir mit Overleaf erstellt (Latex). Im Konzept wurden die einzelnen Bereiche mit ihren Funktionen näher definiert \href{https://www.overleaf.com/project/62b5849dc59dc86e3368e022}{hier}. Teams Call (4h)

\vspace{0.5cm}

\textbf{8.7.2022: 9:30 $\rightarrow$ 11:30} \

Es wurde ein Mockup für die Admin Anwendung erstellt \href{https://github.com/litec-thesis/2223-thesis-5abhit-zoecbe_mayrjo_grupa-cardstorage/blob/main/doc/Mockups/admin_login_mockup.pdf}{hier}. Weiters wurde ein Discord Call mit Patrick durchgeführt, wo wir die Mockups erstellt haben. (2h)

\vspace{0.5cm}

\textbf{11.7.2022: 7:50 $\rightarrow$ 11:30} \

Beginn, mit der Installation von Flutter. Habe mich etwas in Flutter eingearbeitet, \href{https://docs.flutter.dev/get-started/install}{Install Flutter} und \href{https://docs.flutter.dev/get-started/codelab}{Flutter Getting Started}. Danach habe ich mich mit dem groben Layout der Admin App beschäftigt. (3h 40min)

\vspace{0.5cm}

\textbf{11.7.2022: 12:30 $\rightarrow$ 15:45} \

Weiter an der Flutter App gearbeitet. Ich habe mich mit APIs in Flutter beschäftigt. Dazu wurde die Startseite erstellt und mit Test API Daten befüllt. Tutorial \href{https://flutterforyou.com/how-to-fetch-data-from-api-and-show-in-flutter-listview/}{hier}. (3h 15min)

\vspace{0.5cm}

\textbf{12.7.2022: 7:45 $\rightarrow$ 11:45} \

Habe am Startlayout weiter gearbeitet. Habe eine Auswahlbox für die Kartenautomaten hinzugefügt. Weiters wurde das Design leicht verändert und es wurden Icons hinzugefügt. (4h)

\vspace{0.5cm}

\textbf{12.7.2022: 13:10 $\rightarrow$ 15:45} \

Habe an der Kartenansicht weiter gearbeitet. Es ist nun auch funktionell möglich, zwischen allen Karten und nur den Karten eines Tresors zu wechseln. Weiters wurde hinzugefügt, dass es möglich ist, auf eine Karte zu klicken, und sich die Statistik zu jeder Karte anzeigen zu lassen. Aktuell noch mit Testdaten realisiert, da noch keine API vorhanden ist. (2h 35min)

\vspace{0.5cm}

\textbf{13.7.2022: 7:45 $\rightarrow$ 11:30} \

Habe an der Startseite weiter gearbeitet. Habe noch eine Beschreibung hinzugefügt. Weiters habe ich zur App einen Ladebildschirm, ein App Icon und den Namen der App geändert. Weiters habe ich damit begonnen, den Drawer am Homescreen mit Funktionen auszustatten. Es wurde ein Logo eingefügt. Weiters wurden 2 neue Seiten erstellt. Die Einstellungen und App-Info Seite. (3h 45min)

\vspace{0.5cm}

\textbf{13.7.2022: 12:40 $\rightarrow$ 15:15} \

Hab mich damit beschäftigt, einen Dark Mode umzusetzen. Dies stellte sich heraus, dass es nicht so einfach war, aber ich habe es mit etwas Recherche und einem YouTube Tutorial hinbekommen. (2h 35min)

\vspace{0.5cm}

\textbf{26.7.2022: 7:40 $\rightarrow$ 11:50} \

Habe damit begonnen, den Code zu refactoren. Habe eigene Dateien für Farben und Texte erstellt. Weiters habe ich mit der Umsetzung von Dynamischen Titel beschäftigt, habe jedoch gemerkt, das nicht so einfach möglich ist. Arbeite gerade an der Tab bar Leiste um Icons und Text anzeigen zu können, ohne dass ein overflow entsteht. (4h 10min)

\vspace{0.5cm}

\textbf{26.7.2022: 12:35 $\rightarrow$ 14:00} \

Habe weiter am zweiten Tab der App gearbeitet. Dieser behandelt das Hinzufügen und Bearbeiten von Automaten. Dazu habe ich das Layout erstellt und erste Funktionen implementiert. (1h 25min)

\vspace{0.5cm}

\textbf{27.7.2022: 7:25 $\rightarrow$ 12:25} \

Habe heute an der Add Storage Seite gearbeitet. Dazu habe ich die Funktionalität und das Design dafür implementiert. Weiters traten technische Probleme auf, der Android Emulator wollte nicht mehr starten, funktioniert jetzt aber wieder. Weiters habe ich das Git Repo etwas erweitert, die Readmes mit Text befüllt. Danach habe ich mich noch damit beschäftigt, dass die App im Querformat funktioniert, das klappt auch alles einwandfrei. (5h)

\vspace{0.5cm}

\textbf{28.7.2022: 7:25 $\rightarrow$ 12:25} \

Habe an der Card Storage Settings Seite gearbeitet. Die Seite ist soweit fertig. Das komplette Layout inklusive Funktionalität wurden hinzugefügt. Weiters wurden auch die nötigen Vorkehrungen für die API getroffen, das funktioniert so weit. Weiters habe ich noch die Add Card, Add Storage Seite bearbeitet. Dort habe ich auch alles auf die API umgestellt. Habe danach noch etwas Refactoring betrieben. Für jede Seite eine einzelne Text- und Farbdatei erstellt. Weiters habe ich mich noch mit der nativen Ausführung der App unter Windows beschäftigt (nur weil es mich interessiert hat) \href{https://medium.com/flutter-community/flutter-for-desktop-create-and-run-a-desktop-application-ebeb1604f1e0}{hier}. (5h)

\vspace{0.5cm}

\textbf{31.7.2022: 8:15 $\rightarrow$ 11:30} \

Habe an Tab3 weitergearbeitet. Dort habe ich das komplette Layout zum Karten Erstellen und Bearbeiten hinzugefügt. Es wurde auch die komplette Logik implementiert. Weiters habe ich mir Gedanken über Änderungen an der App gemacht. Danach habe ich noch kleine Änderungen vorgenommen und etwas getestet. (3h 15min)

\vspace{0.5cm}

\textbf{1.8.2022: 7:15 $\rightarrow$ 11:30} \

Habe heute an dem User Hinzufügen gearbeitet. Dazu habe ich einen Teil des Layouts und die meisten der Funktionen implementiert. Ich habe ein Suchfeld erstellt, mit dem man sich alle Benutzer anzeigen lassen kann. Es fehlen noch das Design. (4h 15min)

\vspace{0.5cm}

\textbf{8.8.2022: 7:10 $\rightarrow$ 10:10} \

Habe am User Login weitergearbeitet. Habe die Such-Funktion fertig implementiert. Es ist jetzt möglich, Benutzer auszuwählen. Diese werden dann in einer Liste angezeigt. Ein Problem besteht weiterhin. Ich habe keine Lösung gefunden, wie man eine Seite neu laden kann, deshalb werden die Werte nicht direkt angezeigt. Man muss zuerst auf eine andere Seite wechseln, damit diese angezeigt werden. (3h)

\vspace{0.5cm}

\textbf{9.8.2022: 7:05 $\rightarrow$ 10:05} \

Habe damit begonnen, ein GitHub Projekt zu erstellen. Dieses habe ich dann auch mit Inhalt befüllt. Danach habe ich mich mit der rechte Seite beschäftigt. Dazu habe ich begonnen, die GUI zu erstellen und die Logik zu implementieren. Aktuell funktioniert es noch nicht ganz so, wie ich will. (3h)

\vspace{0.5cm}

\textbf{10.8.2022: 6:50 $\rightarrow$ 9:50} \

Habe an der Rights Page weitergearbeitet. Jetzt werden die Werte in der Auswahlbox richtig dargestellt und die ausgewählten Rechte auch angezeigt. Das Reload Problem besteht weiterhin. Habe versucht es mit setState() und dem Valueable Listener zu lösen, aber es hat nicht funktioniert. (3h)

\vspace{0.5cm}

\textbf{15.8.2022: 7:25 $\rightarrow$ 10:25} \

Habe an der rechte Seite weitergearbeitet. Es funktioniert jetzt so weit alles. Das Issue wurde durch einen Refresh Button ``gelöst``. Weiters habe ich am Projekts Board einen weiteren Eintrag hinzugefügt. Es sollte an jeder Seite die Möglichkeit bestehen, dass die ausgewählten Benutzer auch wieder entfernt werden können. (3h)

\vspace{0.5cm}

\textbf{16.8.2022: 7:40 $\rightarrow$ 11:40} \

Habe heute kleine Verbesserungen an der App vorgenommen. Habe mit Herrn Grubauer ein Meeting abgehalten. Wir haben uns über diverse Dinge abgestimmt. Danach haben wir noch die Branch Struktur auf GitHub angepasst. Weiters haben wir noch diverse lokale Änderungen bei Herrn Grubauer vorgenommen. Danach habe ich die App noch etwas getestet. Zum Schluss habe ich noch die Settings Page mit Inhalt befüllt, dazu habe ich den Darkmode Button verschoben. Weiters habe ich noch die App-Info Seite mit etwas Text befüllt. Danach habe ich noch diverse typos in meinen Protokollen behoben und zu jedem Tag die Stundenanzahl hinzugefügt. (4h)

\vspace{0.5cm}

\textbf{17.8.2022: 8:00 $\rightarrow$ 11:10} \

Habe heute an der User Page gearbeitet. Dort habe ich das Design überarbeitet. Danach habe ich ein Meeting mit Herrn Grubauer abgehalten, wo wir klärten,
welche Anforderungen wir an die API haben. (3h 10min)

\vspace{0.5cm}

\textbf{18.8.2022: 7:30 $\rightarrow$ 10:30} \

Habe heute das Problem gelöst, warum der Clear Button auf der Seite 5 und 4 nicht funktionierte. Danach habe ich die Anordnung der Buttons auf den selbigen Seiten geändert. Weiters habe ich noch alle Seiten auf ein einheitliche Design angepasst. Die App wäre so weit auch funktionell, es fehlt lediglich überall die Implementation der API, die noch nicht vorhanden ist. Zukünftige Änderungen sind im GitHub Projekt ersichtlich. (3h)

\vspace{0.5cm}

\textbf{19.8.2022: 7:00 $\rightarrow$ 11:10} \

Habe heute die Remove Seite auf allen Tabs hinzugefügt. Dies funktioniert auch bereits. Danach habe ich noch einen Logout Button hinzugefügt. Weiters wurde
ein Meeting mit Herrn Grubauer durchgeführt, wo wir unser über das Design und Ordner-Struktur abgestimmt haben. Danach habe ich noch die Regex angepasst. (4h 10min)

\vspace{0.5cm}

\textbf{24.8.2022: 14:00 $\rightarrow$ 15:00} \

Habe heute das Github Project um einige Spalten erweitert. Weiters habe ich mir die Protokolle meiner Teammitglieder angeschaut, um über den aktuellen Stand Bescheid zu wissen. Danach habe ich alle Protokoll Files zusammengemerged. (1h)

\vspace{0.5cm}

\textbf{25.8.2022: 7:25 $\rightarrow$ 11:30} \

Habe eine neue Spalte für das Display im Projekt hinzugefügt und mit Inhalt befüllt. Danach habe ich mich wieder einmal mit meinem Refresh Problem beschäftigt. Dazu habe ich einen Call mit Herrn Grubauer durchgeführt, wo wir das Problem mit meinem Refresh gelöst haben. Danach hat er mir noch seinen API Handler erklärt und er hat mir noch gezeigt, wie man das Stunden Zähl Tool von Johannes benutzt. (4h 5min)

\vspace{0.5cm}

\textbf{26.8.2022: 7:15 $\rightarrow$ 11:15} \

Habe heute ich die API von Johannes im Code implementiert. Dies war auch soweit erfolgreich. Danach habe ich mit dem Refactoring vom Code begonnen, wie Herr Grubauer es will. Weiters habe ich mit Herrn Grubauer eine Call durchgeführt, da er das Design wieder ändern will. Zum Schluss habe ich mit dem Refactoring weitergemacht. (4h)

\vspace{0.5cm}

\textbf{29.8.2022: 7:20 $\rightarrow$ 11:20} \

Habe heute die App fertig refactored. Danach habe ich mich mit den warnings beim Starten der App befasst. Die Warnings kamen daher, da seit Flutter 3.0.0 sich intern bei Flutter etwas verändert hat, was die Warnings auslöst. Die Warnings kommen aus der Chart Library, da dort an gewissen Stellen ein !(nullable) steht. Dies wird nicht mehr benötigt, deshalb entstehen diese Warnings. Wenn man die Rufzeichen entfernt, sind die Fehler weg. Danach ich damit begonnen, das Design komplett zu überarbeiten. (4h)

\vspace{0.5cm}

\textbf{30.8.2022: 7:20 $\rightarrow$ 11:20} \

Habe heute am Tab1 weitergearbeitet. Dort habe ich so weit sämtliche Darstellungsfehler behoben. Danach habe ich mit Herrn Grubauer eine kurze Besprechung durchgeführt, um einander auf den aktuellen Stand zu bringen. Weiters habe ich begonnen, den Code von Tab1 zu refactoren, um die Performance zu verbessern. Zum Schluss habe ich die Anwendung auf Fehler getestet. (4h)

\vspace{0.5cm}

\textbf{31.8.2022: 7:15 $\rightarrow$ 11:15} \

Habe heute die Designüberarbeitung fortgesetzt. Es wurde alle Seiten bis auf die Einstellungsseite auf das neue Design geändert. Danach habe ich noch diverse Logikverbesserungen durchgeführt und eine Ladeanimation hinzugefügt. Danach habe ich einen Call mit Herrn Grubauer durchgeführt, wo wir diverse Themen besprochen haben. Zum Schluss habe ich noch kleine Verbesserungen vorgenommen. (4h)

\vspace{0.5cm}

\textbf{31.8.2022: 18:30 $\rightarrow$ 19:30} \

Habe mit Herrn Grubauer versucht, Deep Links zu implementieren. (1h)

\vspace{0.5cm}

\textbf{1.9.2022: 8:00 $\rightarrow$ 12:00} \

Habe heute das Design fertig überarbeitet und einen Themes Switcher implementiert. Der Code muss jetzt noch in mehrere Dateien ausgelagert werden und doppelter Code gelöscht werden. (4h)

\vspace{0.5cm}

\textbf{2.9.2022: 7:15 $\rightarrow$ 11:15} \

Habe heute weiter am Themeswitcher gearbeitet. Dazu habe ich jeden statischen Farbwert auf den Themeswitcher geändert. Der Darkmods funktioniert jetzt auch soweit richtig. Danach habe ich begonnen, den Code in mehrere Dateien aufzuteilen. Die App Bar und der SpeedDialSwitcher wurden bereits ausgelagert. (4h)

\vspace{0.5cm}

\textbf{5.9.2022: 6:45 $\rightarrow$ 12.00} \

Hab heute beim refactoren weitergemacht. Dazu habe ich die Cards Datei in mehrere Dateien aufgeteilt. Danach hat sich ein Problem ergeben. Jedes Mal, wenn man einen Button auf dieser Seite drückt, funktioniert das genau einmal, danach kann man in der App nichts mehr drücken. Das Problem ist, dass ich dir Routen zu den Seiten als Klassenparameter übergebe und deshalb die App nicht mehr weis, wie sie zurückkommt. Deshalb habe ich einen Issue aufgemacht. (5h 15min)

\vspace{0.5cm}

\textbf{6.9.2022: 6:55 $\rightarrow$ 12:00} \

Habe am refactoren weitergemacht. Ich konnte das Problem durch named Routes lösen \href{https://flutter.dev/docs/cookbook/navigation/named-routes}{link}. Danach habe ich weiter refactored. Weiters habe ich mit Herrn Grubauer einen Call durchgeführt. Zum Schluss habe ich noch weiter refactored. (5h 5min)

\vspace{0.5cm}

\textbf{7.9.2022: 7:20 $\rightarrow$ 12:20} \

Habe heute damit begonnen, die App neu zu erstellen, da sie nicht mehr funktioniert hatte. Die App eine Fehlermeldung geworfen, welche sagte, dass die verwendete Android Version zu alt sei. Nach der Neuerstellung funktioniert jetzt wieder alles. Danach habe ich weiter Refactored. Die Suchseite ist jetzt auch fast fertig aufgeteilt. (5h)

\vspace{0.5cm}

\textbf{7.9.2022: 13:30 $\rightarrow$ 15:30} \
Habe Flutter auf dem Laptop installiert und eingerichtet. (2h)

\vspace{0.5cm}

\textbf{8.9.2022: 7:15 $\rightarrow$ 12:15} \
Heute habe ich weiter refactored. Fertig sind bereits der Karten und Storage Teil. Dazu habe ich den Code in mehrere Dateien ausgelagert und Duplikate gelöscht. (5h)

\vspace{0.5cm}

\textbf{8.9.2022: 12:30 $\rightarrow$ 13:30} \
Weiters habe ich noch den User und Statistik Teil überarbeitet. Dazu habe ich den Code in mehrere Dateien ausgelagert und Duplikate gelöscht. (1h)

\vspace{0.5cm}

\textbf{8.9.2022: 14:30 $\rightarrow$ 15:30} \
Später habe ich mit Herrn Grubauer einen Call durchgeführt, wo wir einiges besprochen haben. (30min)

\vspace{0.5cm}
\textbf{16.9.2022: 8:15 $\rightarrow$ 12:30} \

Haben heute in der Stunde die Diplomarbeitsdatenbank mit Inhalt befüllt. Weiters, haben wir damit begonnen, ein Pflichtenhaft zu erstellen. Zum Schluss habe ich mich noch mit GitHub Actions auseinandergesetzt, damit wir Flutter für IOS testen können. (4h 15min)

\vspace{0.5cm}

\textbf{17.9.2022: 8:00 $\rightarrow$ 11:30} \

Wir haben heute das Pflichtenheft erstellt und das Konzept überarbeitet. Dazu haben wir ein Teams Meeting durchgeführt. (3h 30min)

\vspace{0.5cm}

\textbf{19.9.2022: 10:30 $\rightarrow$ 14:30} \

Habe heute die Settings Page erstellt. Weiters habe ich Herr Grubauer bei seinem Problem, dass der Emulator nicht mehr läuft, geholfen. (4h)

\vspace{0.5cm}

\textbf{21.9.2022: 10:10 $\rightarrow$ 12:10} \

Habe heute eine Account-Info Page hinzugefügt. (2h)

\vspace{0.5cm}

\textbf{22.9.2022: 12:30 $\rightarrow$ 14:30} \

Habe heute das Design meiner App an die von Herrn Grubauer angepasst. (2h)

\vspace{0.5cm}

\textbf{26.9.2022: 13:40 $\rightarrow$ 15:15} \

Heute habe ich kleine Design-Änderungen an der App vorgenommen. Danach habe ich mich über den Microsoft Login informiert \href{https://ahmadelshafee28.medium.com/flutter-auth-using-firebase-oauth-and-azure-ad-a6f01b6d584}{(Link)} (1h 45min)

\vspace{0.5cm}

\textbf{30.9.2022: 8:15 $\rightarrow$ 12:00} \

Habe heute kleine Änderungen an der App vorgenommen. Weiters haben wir und Gedanken über den Microsoft Login gemacht. Weiters habe wir das Pflichtenheft verbessert. (3h 45min)

\vspace{0.5cm}

\textbf{30.9.2022: 16:00 $\rightarrow$ 18:00} \

Haben noch das Use Case Diagramm erstellt. (2h)

\vspace{0.5cm}

\textbf{10.10.2022: 8:15 $\rightarrow$ 10:00} \

Habe mich mit Background Notifications befasst. (1h 45min)

\vspace{0.5cm}

\textbf{10.10.2022: 13:45 $\rightarrow$ 15:15} \

Habe einige Beispiele für Notfications mit Firebase probiert, haben aber alle nicht funktioniert. (1h 25min)

\vspace{0.5cm}

\textbf{17.10.2022: 8:20 $\rightarrow$ 11:20} \

Habe begonnen, Notfications mit Firebase einzurichten, damit wir auch Benachrichtigungen senden können, wenn die App geschlossen ist. (3h)

\vspace{0.5cm}

\textbf{17.10.2022: 12:30 $\rightarrow$ 13:30} \

Habe versucht, eine Lösung zu finden, um eine Benachrichtigung zu senden, wenn sich der Flag in der API verändert. Habe aber dazu noch keine Lösung gefunden. (1h)

\vspace{0.5cm}

\textbf{19.11.2022: 9:00 $\rightarrow$ 11:30} \

Habe begonnen, die App auf die API umzustellen und einige Kleinigkeiten behoben. (2h 30min)

\vspace{0.5cm}

\textbf{19.11.2022: 12:30 $\rightarrow$ 16:30} \

Habe das SpeedDial Child entfernt und bin auf einen BottomSheet umgestiegen, damit die Werte dynamisch befüllt werden können. (4h)

\vspace{0.5cm}

\textbf{20.11.2022: 9:30 $\rightarrow$ 11:30} \

Habe heute den Filter fertiggestellt, es besteht aber noch ein Bug bei der Anzeige. Weiters habe ich Methoden zum Post, Put und Delete hinzugefügt. (2h)

\vspace{0.5cm}

\textbf{20.11.2022: 13:00 $\rightarrow$ 15:00} \

Habe den Bug beseitigt, dass bei zu hoher Skalierung ein Overflow eintritt. (2h)

\vspace{0.5cm}

\textbf{21.11.2022: 8:30 $\rightarrow$ 11:30} \

Habe die API Klassen für Storages und Cards erstellt sowie den gesamten Code darauf angepasst. (3h)

\vspace{0.5cm}

\textbf{25.11.2022: 18:30 $\rightarrow$ 20:00} \

Habe heute noch die API soweit fertig eingebunden, dass die Daten von den jeweiligen Karten richtig dargestellt werden. Für die Storages konnte ich es noch nicht testen, da ich die neuste Version der API aktuell nicht starten kann. Weiters habe ich den Code für MQTT implementiert, damit neue Karten angelegt werden können. (1h 30min)

\vspace{0.5cm}

\textbf{28.11.2022: 8:30 $\rightarrow$ 11:30} \

Habe heute einige Fehler beseitigt. (3h)

\vspace{0.5cm}

\textbf{3.12.2022: 9:45 $\rightarrow$ 11:15} \

Wir haben gemeinsam das Plakat erstellt und einige Dinge besprochen (1h 30min)

\vspace{0.5cm}

\textbf{4.12.2022: 8:45 $\rightarrow$ 11:45} \

Habe den Code so angepasst, dass jetzt die richtigen Felder eingelesen werden und an die API übertragen werden. Weiters habe ich einige Fehler behoben, unter anderem bei der Suche, dass Werte doppelt angezeigt wurden. Danach habe ich noch den Code für die Locations implementiert, dass man jetzt beim Storage anlegen auch aus vorhandenen Storage-Locations auswählen kann. (3h)

\vspace{0.5cm}

\textbf{7.12.2022: 12:50 $\rightarrow$ 13:35} \

Diplomarbeitsbesprechung über Schnittstellen (45 min)

\vspace{0.5cm}

\textbf{8.12.2022: 12:30 $\rightarrow$ 16:00} \

Es ist jetzt möglich die Storages beim Karten anlegen und die Location beim Storage anlegen auszuwählen. Weiters kann man jetzt auch neue Locations anlegen. Weiters habe ich einige Probleme bei der Darstellung behoben. Weiters habe ich erste POST versuche auf die API durchgeführt, welche auch funktioniert haben. Hab dann noch die Anforderungen an die API zusammengeschrieben. (3h 30min)

\vspace{0.5cm}

\textbf{9.12.2022: 9:00 $\rightarrow$ 11:45} \

Habe heute die Parental Exceptions, die auf so gut wie allen Seiten vorhanden waren, behoben. Weiters habe ich die Storage und Cards Seiten richtig gestellt, dass auch die richtigen Felder und Werte angezeigt werden. (2h 45min)

\vspace{0.5cm}

\textbf{10.12.2022: 10:00 $\rightarrow$ 12:00}

Habe eine Status Seite für die Storages erstellt, wo ersichtlich ist, ob ein Problem besteht, ob Karten noch nicht zurückgegeben wurden und ob der Storage Online ist. (2h)
\vspace{0.5cm}

\textbf{1.1.2023: 13:00  $\rightarrow$ 13:40}

Meeting mit Johannes über Änderungen, die an der API vorgenommen wurden und welche Sachen jetzt funktionieren. (40min)

\vspace{0.5cm}

\textbf{4.1.2023: 8:45 $\rightarrow$ 11:30}

Habe heute versucht, die neue Version des Containers von der API laufen zu lassen. Es traten aber immer wieder Fehler auf. Diese konnten auch durch die Hilfe von Johannes nicht gelöst werden. Habe dann noch Code für die Websockerts implementiert, um Logs anzuzeigen. Danach habe ich das komplette ID basierte vorgehen im Code, auf den Namen umgestellt, da es keine IDs mehr gibt. Zum Schluss habe ich noch überflüssigen Code gelöscht. (2h 45min)

\vspace{0.5cm}

\textbf{5.1.2023: 10:00 $\rightarrow$ 11:30}

Habe versucht, die neue Version der API zum Laufen zu bekommen. Aber es gestaltete sicher schwieriger als gedacht. Da ich im Verlauf die Idee hatte, dass es am Netzwerke liegen könnte. Das tat es auch, im Hotspot funktioniert alles einwandfrei! (1h 30min)

\vspace{0.5cm}

\textbf{5.1.2023: 12:30 $\rightarrow$ 14:00}

Habe dann diverse Sachen in der App anpassen müssen, damit das meisten Sachen wieder funktionieren. Habe lange an einem Problem gehängt, da immer ein Zertifikatsfehler kam, der sich aber durch generelles Erlauben, von allen und ohne Zertifikaten lösen ließ! (1h 30min)

\vspace{0.5cm}

\textbf{6.1.2023: 10:30 $\rightarrow$ 12:00}

Habe heute die Pink und Focus Funktionen implementiert, welche auch bereits funktionieren. (1h 30min)

\vspace{0.5cm}

\textbf{6.1.2023: 12:30 $\rightarrow$ 17:00} \

Habe die Struktur der Typen Cards, Storages und Ping geändert, damit alle Felder gelesen werden können. Danach habe ich noch die Option, beim Storage hinzugefügt, dass auch Arrays von Karten zugelassen werden. Danach habe ich noch die Status-Seite erstellt, wo man pingen, Focus und Statistiken sehen kann. (4h 30min)

\vspace{0.5cm}

\textbf{9.1.2023: 12:00 $\rightarrow$ 13:30} \

Habe heute einige Probleme in der App behoben. Als erstes, dass die Status-Seite nicht richtig geladen wurde und noch kleiner Fehler behoben. (1h 30min)

\vspace{0.5cm}

\textbf{10.1.2023: 16:00 $\rightarrow$ 20:00} \

Habe heute die Bugs der Status-Seite behoben. Weiters habe ich die Filter-Funktion wieder implementiert. Dort habe ich auch den Bug behoben, dass der gewählte Storage nicht angezeigt wurde. Weiters kann man jetzt wieder Storages / Cards anlegen und bearbeiten. Delete funktioniert noch nicht, da ich den Such – Algorithmus gewechselt habe. (4h)

\vspace{0.5cm}

\textbf{11.1.2023: 17:00 $\rightarrow$ 20:00} \

Habe heute die Methode des Focus von GET auf PUT geändert. Danach habe ich delete Seite der Karten und des Storages fertiggestellt. Weiters habe ich noch einen Refresh Button eingebaut. Zuletzt habe ich den Focus Zustand des Storages noch hinzugefügt. (3h)

\vspace{0.5cm}

\textbf{12.1.2023: 16:30 $\rightarrow$ 18:30} \

Habe heute Reload Buttons auf allen Seiten hinzugefügt. Danach habe ich versucht die WSS zu implementieren, habe es jedoch nicht geschafft. (2h)

\vspace{0.5cm}

\textbf{21.1.2023: 12:45 $\rightarrow$ 14:45} \

Habe heute die Logs Seite fertiggestellt. Die Websockets werden 1-mal bei Start der App verbunden und bleiben dann auch bis zum Schließen verbunden. Weiters werden die Nachrichten über die gesamte Zeit, bis die App geschlossen wird, gespeichert. (2h)

\vspace{0.5cm}

\textbf{22.1.2023: 10:00 $\rightarrow$ 11:30} \

Habe begonnen, die Reservierungen in der App zu implementieren (1h 30min)

\vspace{0.5cm}

\textbf{22.1.2023: 12:30 $\rightarrow$ 15:00} \

Dann bin ich auf ein Problem gestoßen, dass die Reservierungen nicht richtig angezeigt werden. Konnte ich dann durch einen verschachtelten ListView Builder lösen. (2h 30min)

\vspace{0.5cm}

\textbf{23.1.2023: 8:15 $\rightarrow$ 10:00} \

Habe heute eine Bugs behoben, die bei der Darstellung aufgetreten sind. (1h 45min)

\vspace{0.5cm}

\textbf{23.1.2023: 10:30 $\rightarrow$ 12:00} \

Habe begonnen damit, dass man die Zeit, wann man eine Reservierung nicht mehr abholen kann, zu implementieren. Das funktioniert auch so weit (1h 30min)

\vspace{0.5cm}

\textbf{23.1.2023: 12:00 $\rightarrow$ 13:30} \

Danach habe ich noch implementiert, dass man einen User zu einem Admin machen kann. Funktioniert auch so weit. (1h 30min)

\vspace{0.5cm}

\textbf{4.2.2023: 9:00 $\rightarrow$ 12:00} \

Habe heute die Reservierung-Seite angepasst. Es kann jetzt nach Reservierungen gesucht werden. Generell wurde das Layout überarbeitet. Des Weiteren habe ich noch die User-Seite überarbeitet, man bekommt jetzt auch angezeigt, ob der User bereits ein Admin ist. Neu ist auch noch, dass man User jetzt löschen kann. (3h)

\vspace{0.5cm}

\textbf{4.2.2023: 12:15 $\rightarrow$ 15:30} \

Habe auf der Karten-Seite noch eine Suche hinzugefügt. Diese funktioniert auch mit dem Filter. Generell habe ich noch das Design der Seite überarbeitet. Der Verfügbarkeit-Status einer Karte wird jetzt rechts angezeigt. (3h 15min)

\vspace{0.5cm}

\textbf{4.2.2023: 12:15 $\rightarrow$ 15:30} \

Habe auf der Karten-Seite noch eine Suche hinzugefügt. Diese funktioniert auch mit dem Filter. Generell habe ich noch das Design der Seite überarbeitet. Der Verfügbarkeit-Status einer Karte wird jetzt rechts angezeigt. (3h 15min)

\vspace{0.5cm}

\textbf{5.2.2023: 11:00 $\rightarrow$ 13:00} \

Habe heute mit der Dokumentation begonnen. Dazu habe ich mir eine grobe Struktur überlegt, was alles enthalten sein soll. Als Erstes werde ich generelle Dinge zum Admin-Login erklären. Danach wird die Implementierung folgen. Weiters werde ich dann noch erklären, falls die Diplomarbeit in Zukunft noch weiter entwickelt werden soll, wie man die Arbeitsumgebung erstellt. Zum Schluss werde ich die Installation der App erklären und das Benutzerhandbuch erstellen. (2h)

\vspace{0.5cm}

\textbf{5.2.2023: 12:30 $\rightarrow$ 15:40} \

Habe danach mit der Dokumentation begonnen. Dazu habe ich als Erstes die Theoretischen Grundlagen formuliert. Welche Aufgaben der Admin Login hat, welche Benutzer Zugriff haben, wie man sich anmeldet, die Navigation in der App, welche Funktionen die App bietet und ein Ablauf Diagramm erstellt. (3h 10min)

\vspace{0.5cm}

\textbf{5.2.2023: 16:00 $\rightarrow$ 17:30} \

Habe mit Herrn Grubauer die Struktur der Dokumentation besprochen, welche Sachen wir gemeinsam erklären sollten. (1h 30min)

\vspace{0.5cm}

\textbf{6.2.2023: 8:20 $\rightarrow$ 12:00} \

Habe damit begonnen, die API Authentifizierung zu implementieren. Stellte sich als schwieriger als Gedacht heraus. Habe das mit einem SecureStorage gelöst, in dem die Tokens gespeichert werden. Danach habe ich mich noch damit befasst, wenn ein Token abläuft, dass ein neuer generiert wird. (3h 40min)

\vspace{0.5cm}

\textbf{6.2.2023: 12:30 $\rightarrow$ 15:30} \

Weiters habe ich dann noch den Token bei den Websockets implementiert. Zum Schluss habe ich dann noch das Design der Storage Seite, auf das der Karten-Seite angepasst. (3h)

\vspace{0.5cm}

\textbf{6.2.2023: 17:00 $\rightarrow$ 19:30}

Habe heute noch das Design auf allen Seiten angeglichen. Des Weiteren habe ich einige kleine Fehler behoben und ein paar Tests durchgeführt. (2h 30min)

\vspace{0.5cm}

\textbf{7.2.2023: 17:30 $\rightarrow$ 19:00}

Habe an der Dokumentation weitergeschrieben. Habe die Technologien der Admin App verfasst und begonnen, die Implementierung zu verfassen. (1h 30min)

\vspace{0.5cm}

\textbf{8.2.2023: 15:30 $\rightarrow$ 20:30}

Habe heute Error Handling auf der Storage Seite hinzugefügt. Es wird überprüft, ob der jeweilige Storage bereits vorhanden ist und dann angezeigt. Wenn bei der Übertragung an die DB etwas schiefgeht, wird dies auch angezeigt. (5h)

\vspace{0.5cm}

\textbf{9.2.2023: 17:00 $\rightarrow$ 20:15}

Habe heute als Erstes den Default User Option hinzugefügt, dass bei jedem Storage ein User für Patrick angelegt wird. Danach habe ich das Error Handling erweitert. Es werden jetzt praktisch alle Fehler in irgendeiner Weise gehandelt und dem User angezeigt, z. B.: Es können keine Felder beim Anlegen leer gelassen werden, es wird geprüft, dass der Storage / Karten Name noch nicht existiert und ob gegenüber der API ein Fehler aufgetreten ist. Dazu werden die API Error Codes ausgewertet ($200 = \text{OK}$, $400 = \text{Error}$, $401 = \text{Ungültiger Token}$). (3h 10min)

\vspace{0.5cm}

\textbf{10.2.2023: 16:00 $\rightarrow$ 18:15}

Habe mit Herrn Grubauer unsere 2 Apps gemerged. Diese funktionieren jetzt, es fehlt lediglich der Switch-Button zwischen den Apps. (2h 15min)

\vspace{0.5cm}

\textbf{11.2.2023: 9:00 $\rightarrow$ 13:00} \

Habe heute weiter an der Doku gearbeitet. Dazu habe ich die allgemeinen Technologien der beiden Apps noch einmal überarbeitet. Dazu habe ich noch mit Patrick einige Calls durchgeführt, um uns abzustimmen. (4h)

\vspace{0.5cm}

\textbf{12.2.2023: 8:00 $\rightarrow$ 12:00} \

Habe heute damit begonnen. einen Timer beim Ausborgen der Karten einzufügen. Das hat sich anfänglich als schwierig gestaltet, da ich einen Timer verwendete, welcher praktisch nicht möglich war zu unterbrechen. Danach hatte ich nach einem anderen Package gesucht, mit dem es dann sofort funktioniert hat. (4h)

\vspace{0.5cm}

\textbf{12.2.2023: 12:30 $\rightarrow$ 18:30} \

Später habe ich dann damit begonnen, die Karten-Seiten zu überarbeiten. Dazu habe ich die 4 vorhandenen Files weiter aufgeteilt, damit Duplikationen entfernt werden konnten. Somit konnte ich etwa 400 Zeilen von 1200 Zeilen Code einsparen. Dazu habe ich dann auch noch kleine Änderungen am Design vorgenommen. Es wird bei den Karten jetzt auch der Storage Name angezeigt und statt True und False bei der Verfügbarkeit, Ja und Nein. Des Weiteren habe ich das Error Handling noch etwas verbessert. (6h)

\vspace{0.5cm}

\textbf{12.2.2023: 18:30 $\rightarrow$ 19:30} \

Habe bei der Doku den theoretischen Teil der Admin Sicht, noch etwas genauer zu beschreiben. (1h)

\vspace{0.5cm}

\textbf{13.2.2023: 8:00 $\rightarrow$ 11:30} \

Habe heute den theoretischen Teil fertig überarbeitet. (3h 30min)

\vspace{0.5cm}

\textbf{13.2.2023: 12:40 $\rightarrow$ 15:40} \

Weiters habe ich dann noch das Ändern des App Names, Icons und des Ladebildschirms beschrieben. Danach habe ich mit dem Erklären der Implementation der Admin Sicht begonnen. (3h)

\vspace{0.5cm}

\textbf{14.2.2023: 18:00 $\rightarrow$ 20:30} \

Habe heute wieder weiter an der Doku geschrieben. Dazu habe ich mit der Weiterentwicklung begonnen. (2h 30min)

\vspace{0.5cm}

\textbf{15.2.2023: 16:40 $\rightarrow$ 18:40} \

Habe heute die Weiterentwicklung fertig geschrieben. Danach habe ich mit dem Benutzer Handbuch begonnen. (2h)

\vspace{0.5cm}

\textbf{17.2.2023: 7:00 $\rightarrow$ 11:50} \

Habe an der Doku weiter geschrieben. Benutzer Handbuch fertiggestellt. (4h 50min)

\vspace{0.5cm}

\textbf{17.2.2023: 12:30 $\rightarrow$ 14:00} \

Habe die allgemeinen Sachen erstellt und noch kleinere andere Sachen bearbeitet. (3h 30min)

\vspace{0.5cm}

\textbf{18.2.2023: 8:00 $\rightarrow$ 11:55} \

Habe heute Kleinigkeiten angepasst und Rechtschreibfehler behoben. (3h 55min)

\vspace{0.5cm}

\textbf{18.2.2023: 12:30 $\rightarrow$ 15:30} \

Habe meine Teil der Doku fertiggestellt, dazu habe ich die Protokolle noch hinzugefügt und das Pflichtenheft. (3h)

\vspace{0.5cm}

\textbf{19.2.2023: 8:30 $\rightarrow$ 12:30} \

Habe heute im Code die Middleware implementiert. Es funktioniert auch soweit alles, aber es tritt beim Karten anlegen ein Parental Error auf, welcher noch gefixt werden muss. (4h)

\vspace{0.5cm}

\textbf{20.2.2023: 9:00 $\rightarrow$ 13:30} \

Habe heute die Websockets in eine eigene Datei ausgelagert. Des Weiteren funktioniert jetzt auch, wenn eine neue Karte angelegt wird, dass der WS überprüft wird. (4h 30min)

\vspace{0.5cm}

\textbf{20.2.2023: 18:40 $\rightarrow$ 22:40} \

Haben die Doku mit Johannes gemerged. Die Doku ist so weit auch fertig. (3h 20min)

\vspace{0.5cm}

\textbf{22.2.2023: 9:00 $\rightarrow$ 13:30} \

Refactored, Storage und Karten-Seite. (4h 30min)

\vspace{0.5cm}

\textbf{24.2.2023: 8:30 $\rightarrow$ 12:30} \

Refactored, Status-Seite. Bugs behoben. (4h)

\vspace{0.5cm}

\textbf{24.2.2023: 13:00 $\rightarrow$ 16:00} \

Habe bei der Doku einige Sachen korrigiert. Use Case erstellt.(3h)

\vspace{0.5cm}

\textbf{25.2.2023: 8:30 $\rightarrow$ 12:00} \

Refactored User Page. (3h 30min)

\vspace{0.5cm}

\textbf{25.2.2023: 12:30 $\rightarrow$ 18:00} \

Refactored Reservation Page. (5h 30min)

\vspace{0.5cm}

\textbf{26.2.2023: 9:00 $\rightarrow$ 14:00} \

Habe heute fertig Refactored. Danach habe ich die App so umgebaut, dass sich der Inhalt dynamisch an die Größe des Display anpasst. Jetzt fehlt nur noch der Log-Out Button und der Button, zum Wechseln zur Client Sicht. (5h)

\vspace{0.5cm}

\textbf{27.2.2023: 8:20 $\rightarrow$ 12:30} \

Log-Out Button, wechsel zur Client App hinzugefügt. User kann jetzt auch wieder Admin werden. (4h 10min)

\vspace{0.5cm}

\textbf{27.2.2023: 13:45 $\rightarrow$ 15:25} \

Refactored Middleware und Theme Provider. (1h 30min)

\vspace{0.5cm}

\textbf{6.3.2023: 8:00 $\rightarrow$ 11:30} \

Habe heute noch das App Icon sowie den App Namen gepasst. Danach habe ich den Ladebildschirm noch eingefügt. Des Weiteren habe ich den letzten großen Bug behoben und die ganzen Files in deren Ordner geschoben. Die Admin Sicht ist nun endlich fertig!(3h 30min)

\vspace{0.5cm}

\textbf{10.3.2023: 12:00 $\rightarrow$ 21:00} \

Doku überarbeitet, Folder, Video und PowerPoint fertiggestellt. (9h)