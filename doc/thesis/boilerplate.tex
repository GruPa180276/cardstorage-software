% https://stackoverflow.com/questions/56301839/signature-page-in-latex
\newcommand\signature[1]{% Name
\begin{center}
\begin{minipage}{10cm}
    \centering
    \vspace{3cm}\par
    \noindent
    \hspace{1.25cm}\rule{10cm}{0.5pt}\par
    \noindent
    \hspace{2.5cm}\textbf{#1}
\end{minipage}
\end{center}}

% \section{Sperrvermerk}
% \lipsum[1]

\section{Haftungsausschluss}
Die Inhalte dieser Diplomarbeit wurden nach bestem Wissen und Gewissen erarbeitet. Es kann jedoch keine, wie immer geartete Verantwortung oder Haftung für deren Aktualität, Vollständigkeit oder Richtigkeit übernommen werden. Es handelt sich bei dieser Diplomarbeit im Übrigen keinesfalls um eine sogenannte ``Ingenieursbüroarbeit''.

\newpage

\section{Eidesstattliche Erklärung}
Wir erklären Eides statt, dass wir die vorliegende Diplomarbeit selbstständig und ohne fremde Hilfe verfasst, andere als die angegebenen Quellen und Hilfsmittel nicht verwendet und die den benutzten Quellen inhaltlich beziehungsweise wörtlich entnommenen Stellen als solche kenntlich gemacht haben.

\vspace{1.5cm}
\noindent
\textbf{Linz, im März 2023}

\begin{tikzpicture}[remember picture,overlay]\node[xshift=9.5cm,yshift=-2.26cm]{\includegraphics[width=.45\textwidth]{MJ/assets/signature-gp.png}};\end{tikzpicture}
\signature{\titlePageFullNameGp}

\begin{tikzpicture}[remember picture,overlay]\node[xshift=9.35cm,yshift=-2.36cm]{\includegraphics[width=.45\textwidth]{MJ/assets/signature.png}};\end{tikzpicture}
\signature{\titlePageFullNameMj}

\begin{tikzpicture}[remember picture,overlay]\node[xshift=9.5cm,yshift=-2.46cm]{\includegraphics[width=.45\textwidth]{MJ/assets/signature-zb.png}};\end{tikzpicture}
\signature{\titlePageFullNameZb}

\newpage

\section{Danksagung}
In diesem Zusammenhang möchten wir uns bei allen Lehrern und Familienmitgliedern herzlich bedanken, die uns während der stressigen Abschlussarbeit begleitet haben. Besonders möchten wir uns bei unserer Betreuerin, Frau Prof. Dr. Susanne Hofer, bedanken, die uns in dieser herausfordernden Zeit tatkräftig unterstützt und wertvolle Hinweise gegeben hat.

\newpage

\section{Gender-Hinweis}
In der vorliegenden Arbeit werden geschlechterinklusive Formulierungen verwendet, um alle Geschlechter einzuschließen. Bei Personenbezeichnungen wird, soweit möglich, eine geschlechtsneutrale Formulierung verwendet. Falls dies nicht möglich ist, werden beide Geschlechter genannt oder abwechselnd die weibliche und männliche Form verwendet. Wir möchten betonen, dass der Begriff ``Personen'' oder ``Menschen'' in dieser Arbeit geschlechterunabhängig zu verstehen ist.

\newpage

\section{Einleitung}
Das Ausborgen einer Karte, um im Schularbreitenmodus drucken zu können, erwies sich als zeitaufwändiger Prozess mit einigen Herausforderungen. Die Karten wurden in einer Mappe aufbewahrt, die mit einer Liste aller verfügbaren Karten versehen war. Das Ausborgen einer Karte, um im Schularbreitenmodus drucken zu können, stellte sich aufgrund der Lage der Kartenmappe als eine Herausforderung dar. Obwohl die Anzahl der Karten überschaubar und ihre Auffindbarkeit einfach ist, kann eine Lehrkraft Schwierigkeiten haben, eine Karte auszuleihen oder zurückzugeben, wenn der Raum nicht durchgängig besetzt ist. In solchen Fällen kann es zu Wartezeiten kommen oder alternative Formen zur Rückgabe (z.B. das Postfach einer Lehrkraft im LIZ) müssen genutzt werden. Das Ausleihen einer Karte erfordert daher Zeit und Geduld.\bigskip

\noindent
Die vorliegende Diplomarbeit widmet sich dem Thema der Digitalisierung und der Optimierung des Ausleih-Prozesses von Karten.  Diese Aufgaben umfassen die Erstellung von Backend-Systemen, APIs und der Frontend-Entwicklung einer Flutter-Anwendung. Im Rahmen dieses Projekts wurde eine interdisziplinäre Zusammenarbeit mit der Abteilung für Mechatronik initiiert, um die Entwicklung von hardwarebasierten Aufgaben voranzutreiben. 


Im Rahmen einer weiteren Diplomarbeit arbeiten die Mechatroniker an der Konstruktion
eines Tresors, der als Hauptaufgabe die sichere Aufbewahrung der Karten gewährleistet. Die Kombination der beiden Diplomarbeiten soll in Zukunft einen orts- und zeitungebundenen Zugriff auf die Druckerkarten ermöglichen und den Ausleihprozess automatisieren.





\newpage
\section{Abstract}
Borrowing a card to print in school-wide mode proved to be a time-consuming process with some challenges. The cards were kept in a folder with a list of all available cards. Borrowing a card to print in school-wide mode proved to be a challenge due to the location of the card folder. Although the number of cards is manageable and they are easy to locate, a teacher may have difficulty borrowing or returning a card if the room is not consistently occupied. In such cases, there may be waiting times or alternative forms of return (such as using a teacher's mailbox in the LIZ) may need to be utilized. Borrowing a card, therefore, requires time and patience.\bigskip

\noindent
The present thesis is dedicated to the topic of digitization and optimization of the card lending process. These tasks include the creation of backend systems, APIs, and frontend development of a Flutter application. As part of this project, an interdisciplinary collaboration with the Department of Mechatronics was initiated to advance the development of hardware-based tasks.

As part of another thesis, the mechatronics engineers are working on the construction of a safe that ensures the secure storage of the cards. The combination of both theses is expected to enable location- and time-independent access to the printing cards in the future and to automate the lending process.