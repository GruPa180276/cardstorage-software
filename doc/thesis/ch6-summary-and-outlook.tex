\section{Zusammenfassung}
Die Diplomarbeit widmete sich der Entwicklung einer Software, die in der Lage sein sollte, die Schließfächer der Mechatroniker anzusteuern und mit ihnen zu interagieren. Dabei war es von größter Bedeutung eine hohe Benutzerfreundlichkeit zu gewährleisten, um sicherzustellen, dass die Anwendung von den Nutzern einfach und intuitiv bedient werden kann. 

Um eine schnelle und effiziente Bereitstellung des Backends zu ermöglichen, wurde ein Docker-Container eingesetzt. Dadurch war es möglich, die Anwendung rasch und einfach auf verschiedenen Systemen zu installieren und auszuführen.

Darüber hinaus wurde eine native Anwendung in Flutter entwickelt, die auf jeder Plattform ausgeführt werden kann. Dies bietet den Nutzern maximale Flexibilität bei der Verwendung der Anwendung und gewährleistet eine nahtlose Integration in die bestehende Infrastruktur der Mechatroniker.

Die Diplomarbeit zeigt somit auf, wie durch den Einsatz moderner Technologien und Methoden eine effektive und benutzerfreundliche Softwarelösung für den Einsatz in der Industrie entwickelt werden kann.

\section{Bewertung}

Die Diplomarbeit zeigt eine beeindruckende Leistung bei der Entwicklung einer Softwarelösung, die nahtlos in den Gesamtprozess der Mechatronik-Diplomarbeit integriert werden kann. Insbesondere die einfache Verbindung mit dem Tresor, sobald dieser fertiggestellt ist, ermöglicht eine reibungslose Zusammenarbeit zwischen verschiedenen Abteilungen und erleichtert die Aufgaben der IT-Abteilung.

Die Diplomarbeit präsentiert den Einsatz moderner Technologien wie Docker-Containern und Flutter, um innovative Lösungen zu implementieren. Es wird gezeigt, dass ein Verständnis für aktuelle Trends in der Softwareentwicklung vorhanden ist und eine hohe technische Kompetenz sowie eine kreative Herangehensweise an die Softwareentwicklung demonstriert wird.

Die Diplomarbeit hat somit bewiesen, dass moderne Technologien und praxisorientierte Herangehensweisen einen wesentlichen Beitrag zur Verbesserung von Arbeitsabläufen leisten können. Diese erworbenen Fähigkeiten und Kenntnisse können auch in zukünftigen Projekten von Nutzen sein und zeigen das Potenzial einer erfolgreichen Diplomarbeit auf.

\section{Ausblick}
Mit der erfolgreichen Entwicklung und Implementierung unserer Softwarelösung haben wir gezeigt, dass moderne Technologien und Methoden eine wichtige Rolle bei der Verbesserung von industriellen Prozessen spielen können. Es ist wichtig zu betonen, dass bei der Entwicklung unserer Softwarelösung eine effektive Trennung von Frontend und Backend durch den Einsatz einer Middleware erreicht wurde. Dadurch wurde eine weitgehend unabhängige Bearbeitung ermöglicht und eine nahtlose Integration mit zukünftigen Lösungen erleichtert.

Darüber hinaus sind wir davon überzeugt, dass die Erkenntnisse und Erfahrungen, die wir während dieses Projekts gesammelt haben, auch bei zukünftigen Projekten von Nutzen sein werden. Wir sind bereit, unser Wissen und unsere Fähigkeiten einzusetzen, um weitere Herausforderungen im Bereich der Softwareentwicklung in der Industrie anzugehen und gemeinsam mit Kunden optimale Lösungen zu erarbeiten.

Insgesamt haben wir uns zum Ziel gesetzt, kontinuierlich unsere Fähigkeiten und Kenntnisse zu erweitern. Wir sind motiviert und bereit, auch zukünftig innovative und effektive Lösungen für die Herausforderungen in der Industrie zu entwickeln.
